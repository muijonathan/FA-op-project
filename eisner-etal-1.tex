\subsection{Exercises for chapter 1}

\begin{exercise}
	\label{exer:torus-cesaro}
	For $\lambda \in \mathbb{T}$, prove that
	\begin{equation*}
		\frac{1}{n} \sum_{j=1}^n \lambda^j \text{ converges as } n \to \infty
	\end{equation*}
	and calculate the limit. Is the convergence uniform in $\lambda \in \mathbb{T}$?
\end{exercise}

\begin{proof}[Solution]
	Let $S_n(\lambda) := \frac{1}{n}\sum_{j=1}^n \lambda^j$. If $\lambda=1$, then clearly $S_n(1) = 1$ for all $n\in\NN$. For $\lambda \ne 1$, we have
	\begin{equation*}
		|S_n(\lambda)| = \frac{|\lambda|}{n} \left|\sum_{j=0}^{n-1} \lambda^j \right| = \frac{|\lambda|}{n} \frac{|\lambda^n - 1|}{|\lambda - 1|} \le \frac{2}{n} \frac{|\lambda|}{|\lambda-1|} \longrightarrow 0
	\end{equation*}
	as $n\to\infty$. Hence
	\begin{equation*}
		\lim_{n \to \infty} S_n(\lambda) = \begin{cases}
			0 \qquad \lambda \ne 1 \\
			1 \qquad \lambda = 1,
		\end{cases}
	\end{equation*}
	and since the limit function is discontinuous, the convergence is not uniform.
\end{proof}

\begin{exercise}
	Prove that a convergent sequence $(a_n) \subset \CC$ with limit $a$ is \emph{Ces\`{a}ro convergent} to the same limit, i.e.\ $A_n := \frac{1}{n}\sum_{j=1}^n a_j$ converges to $a$ as $n\to\infty$.
\end{exercise}

\begin{proof}[Solution]
	Let $\varepsilon >0$ be arbitrary. We show that there exists $N\in\NN$ such that $|A_{N+k}-a|$ can be made as small as desired for sufficiently large $k\in\NN$. Observe that
	\begin{align*}
		|A_{N+k}-a| = \left| \frac{1}{N+k}\sum_{j=1}^{N+k}a_j - a\right| &= \left|\frac{1}{N+k}\sum_{j=1}^N (a_j-a) + \frac{1}{N+k}\sum_{j=N+1}^{N+k}(a_j - a)\right| \\
		&\le \frac{M_N}{N+k} + \frac{1}{N+k}\sum_{j=N+1}^{N+k}|a_j-a|,
	\end{align*}
	where $M_N = |\sum_{j=1}^N (a_j-a)|$. Choose $N\in\NN$ such that $|a_j-a| < \varepsilon$ for all $j \ge N$. If $k_0\in\NN$ is chosen large enough to satisfy $M_N \le (N+k_0)\varepsilon$, then
	\begin{equation*}
		|A_{N+k}-a| \le \frac{M_N}{N+k} + \frac{1}{N+k}\sum_{j=N+1}^{N+k}|a_j-a| \le \varepsilon + \frac{k}{N+k} \varepsilon \le 2\varepsilon
	\end{equation*}
	for all $k\ge k_0$. Hence $A_n \to a$ as $n\to\infty$.
\end{proof}

\begin{exercise}
	Give an example of a non-convergent sequence in $\CC$ that is Ces\`{a}ro convergent. Give an example of a bounded sequence that is not Ces\`{a}ro convergent.
\end{exercise}

\begin{proof}[Solution]
	The sequence $a_n = (-1)^n$ is non-convergent, but is Ces\`{a}ro convergent to 0 (this is easily verified directly, or can be seen as a special case of Exercise~\ref{exer:torus-cesaro}).
\end{proof}

\begin{exercise}
	Let $T$ be a $d\times d$ permutation matrix. Prove that
	\begin{equation*}
		\frac{1}{n}\sum_{j=1}^n T^j \text{ converges as } n \to \infty
	\end{equation*}
	and determine the limit.
\end{exercise}

\begin{exercise}
	Let $T$ be a $d\times d$ complex matrix. Prove that $\sup_{n\in\NN} \|T^n\| < \infty$ (i.e.\ the matrix is \emph{power-bounded}) if and only if
	\begin{equation*}
		A_n := \frac{1}{n}\sum_{j=1}^n T^j \text{ converges as } n\to\infty.
	\end{equation*}
	[\emph{Hint}: consider the Jordan normal form of $T$].
\end{exercise}

\subsection{Notes and exercises for chapter 2}
\textbf{Convention}: in this book, a \emph{topological dynamical system} is a pair $(K;\varphi)$, where $K$ is a non-empty compact Hausdorff space, and $\varphi : K \to K$ is continuous.

\begin{definition}
	A subset $A \subseteq K$ is called \emph{invariant} if $\varphi(A) \subseteq A$, \emph{stable} if $\varphi(A) = A$, and \emph{bi-invariant} if $\varphi^{-1}(A) = A$.
\end{definition}

\begin{lemma}[Exercise 2.9]
	Let $(K;\varphi)$ be a topological dynamical system, and let $A \subseteq K$. The following assertions hold:
	\begin{enumerate}[\upshape (i)]
		\item If $A$ is bi-invariant or stable, then $A$ is invariant.
		\item If $A$ is stable and $\varphi$ injective, then $A$ is bi-invariant.
		\item If $A$ is bi-invariant and $\varphi$ surjective, then $A$ is stable.
	\end{enumerate}
\end{lemma}

\begin{proof}
	The proof essentially follows from basic facts about mappings. Let $X, Y$ be non-empty sets and $f:X \to Y$. Then
	\begin{itemize}
		\item For all $A \subseteq X$, $A \subseteq f^{-1}(f(A))$ with equality if and only if $f$ is injective.
		\item For all $B \subseteq Y$, $f(f^{-1}(B)) \subseteq B$ with equality if and only if $f$ is surjective.
	\end{itemize}
	
	(i): Stable trivially implies invariant. If $\varphi^{-1}(A) = A$, then $A \supseteq \varphi(\varphi^{-1}(A)) = \varphi(A)$, so $A$ is invariant.
	
	(ii): Since $\varphi(A)=A$ and $\varphi$ is injective, we obtain $\varphi^{-1}(A) = \varphi^{-1}(\varphi(A)) = A$.
	
	(iii): Since $\varphi^{-1}(A)=A$ and $\varphi$ is surjective, we obtain $\varphi(A) = \varphi(\varphi^{-1}(A)) = A$.
\end{proof}

\begin{lemma}
	Suppose that $(K;\varphi)$ is a topological dynamical system, and $A \subseteq K$ is a non-empty, closed and invariant set. Then there is a non-empty closed set $B \subseteq A$ such that $\varphi(B)=B$, i.e.\ $B$ is stable.
\end{lemma}

\begin{proof}
	Since $K$ is a compact Hausdorff space, $A$ is also compact. Then
	\begin{equation*}
		A \supseteq \varphi(A) \supseteq \varphi^2(A) \supseteq \cdots \supseteq \varphi^n(A)
	\end{equation*}
	holds for all $n\in\NN$, and each of these sets is non-empty and compact, since $\varphi$ is continuous. The collection of closed sets $\{\varphi^n(A)\}_{n\in\NN_0}$ has the finite intersection property, and hence, by the compactness of $K$, the set $B = \bigcap_{n\in\NN} \varphi^n(A)$ is non-empty and compact. Clearly $\varphi(B) \subseteq B$. To prove the reverse inclusion, let $x\in B$ be arbitrary. Then for each $n\in\NN$, we have $x \in \varphi^n(A)$, in other words, $\varphi^{-1}\{x\} \cap \varphi^{n-1}(A) \ne \emptyset$ for all $n\in\NN$. Therefore $\varphi^{-1}\{x\} \cap B \ne \emptyset$, i.e.\ $x\in \varphi(B)$. This completes the proof. 
\end{proof}

\begin{corollary}
	For a topological system $(K;\varphi)$, define
	\begin{equation*}
		K_s := \bigcap_{n\in\NN_0} \varphi^n(K).
	\end{equation*}
	Then $(K_s;\varphi)$ is the unique maximal, hence largest, surjective subsystem of $(K;\varphi)$.
\end{corollary}

\begin{exercise}[Ex.\ 2.11]
	Let $(K;\varphi)$ be a topological system. Show that:
	\begin{enumerate}[(i)]
		\item If $A \subseteq K$ is invariant (resp.\ stable), then $\overline{A}$ is invariant (resp.\ stable).
		\item The intersection of arbitrarily many (bi-)invariant subsets of $K$ is again (bi-)invariant.
		\item If $\Psi:(K;\varphi) \to (L;\psi)$ is a homomorphism of topological systems and $A \subseteq K$ is invariant (resp.\ stable), then so is $\Psi(A) \subseteq L$.
	\end{enumerate}
\end{exercise}

\begin{proof}
	(i): If $A$ is invariant, then $\overline{\varphi(A)} \subseteq \bar{A}$. By continuity of $\varphi$, the set $\varphi^{-1}(\overline{\varphi(A)})$ is closed and clearly contains $A$. Hence $\bar{A} \subseteq \varphi^{-1}(\overline{\varphi(A)})$, which implies $\varphi(\bar{A}) \subseteq \overline{\varphi(A)} \subseteq \bar{A}$.
	
	Now assume $A$ is stable, i.e.\ $\varphi(A)=A$.  Since $K$ is a compact Hausdorff space, every continuous map on $K$ maps closed sets to closed sets. Thus $\varphi(\bar{A})$ is a closed set containing $\varphi(A)=A$, which implies $\overline{\varphi(A)} \subseteq \varphi(\bar{A})$. However, by the above argument, the reverse inclusion also holds, hence $\varphi(\bar{A}) = \bar{A}$.
	
	(ii): This is an immediate consequence of the following general fact: for any mapping $f:K \to K$ and any collection of subsets $\{A_i\}_{i \in I}$ of $K$, it holds that $f^{-1}(\bigcap_{i\in I} A_i) = \bigcap_{i\in I} f^{-1}(A_i)$ and $f(\bigcap_{i\in I} A_i) = \bigcap_{i\in I} f(A_i)$.
	
	(iii): We have $\psi \circ \Psi = \Psi \circ \varphi$. If $A \subseteq K$ is invariant (resp.\ stable), then $\psi(\Psi(A)) = \Psi(\varphi(A)) \subseteq \Psi(A)$, respectively $=\Psi(A)$.
\end{proof}

\subsection{Notes and exercises for chapter 3}

A topological dynamical system is \emph{minimal} if there are no non-trivial, $\varphi$-invariant closed subsets. In other words, if $A \subseteq K$ is closed and invariant, then either $A=\emptyset$ or $A=K$.

\begin{exercise}[Exercise 3.1]
	Prove that a minimal subsystem of an invertible topological system is itself invertible.
\end{exercise}

\begin{proof}
	We show that $\varphi^{-1}(A) = A$. Since $\varphi$ is injective, by Lemma~\ref{lem:invariant-stable}(ii) it suffices to show that $A$ is stable under $\varphi$. By Lemma~\ref{lem:stable-subset}, $A$ contains a non-empty, closed, stable subset $B$. However, if $B$ is a strict subset of $A$, this would contradict the minimality of $A$. Hence $B=A$ is stable.
\end{proof}

\begin{proposition}[Proposition 3.3 / Exercise 3.4]
	For a topological system $(K;\varphi)$, the following are equivalent:
	\begin{enumerate}[\upshape (i)]
		\item $(K;\varphi)$ is minimal.
		\item $\mathrm{orb}_+(x)$ is dense in $K$ for each $x\in K$.
		\item $K = \bigcup_{n\in\NN_0} \varphi^{-n}(U)$ for every non-empty open set $U \subseteq K$.
	\end{enumerate}
	In particular, every minimal system is forward transitive.
\end{proposition}

\begin{proof}
	`(i) $\Rightarrow$ (ii)': Assume that $K$ is minimal, and choose any $x\in K$. Then $A = \overline{\mathrm{orb}_+(x)}$ is a closed, $\varphi$-invariant subset (see Exercise~\ref{exer:2-11}), so $A=K$ by minimality.
	
	`(ii) $\Rightarrow$ (iii)': Let $U \subseteq K$ be non-empty and open. By the continuity of $\varphi$, note that each $\varphi^{-n}(U)$ is open. Fix an arbitrary $x\in K$. By assumption, $\mathrm{orb}_+(x)$ is dense in $K$, and hence has non-empty intersection with some $\varphi^{-n}(U)$.
	
	`(iii) $\Rightarrow$ (i)': Assume for contradiction that there exists a non-empty, closed invariant set $A$ with $A \ne K$. Then the complement $A^c$ is a non-empty open set, and $K = \bigcup_{n\in\NN_0} \varphi^{-n}(A^c)$ by assumption. Note that $A \subseteq \varphi^{-n}(A)$ for all $n\in\NN_0$ (by invariance), and $\emptyset = \varphi^{-n}(A \cap A^c) = \varphi^{-n}(A) \cap \varphi^{-n}(A^c)$. Thus $\varphi^{-n}(A^c)$ does not intersect $A$ for all $n\in\NN_0$, and it follows that $K = \bigcup_{n\in\NN_0} \varphi^{-n}(A^c)$ does not intersect $A$, an absurdity.
	
	\emph{Remark}: note that compactness of $K$ was not used in the proof.
\end{proof}

\begin{exercise}[Exercise 3.6]
	Let $x_0$ be a recurrent point of $(K;\varphi)$. Show that $x_0$ is \emph{infinitely recurrent}, i.e.\ for every neighbourhood $U$ of $x_0$, one has
	\begin{equation*}
		x_0 \in \limsup_n \varphi^{-n}(U) := \bigcap_{n\in\NN} \bigcup_{m\ge n} \varphi^{-m}(U).
	\end{equation*}
\end{exercise}

\begin{proof}
	Fix a neighbourhood $U$ of $x_0$. Since $x_0$ is recurrent, there exists $n_0\in\NN$ such that $\varphi^{n_0}(x_0) \in U$, or equivalently $x_0 \in \varphi^{-n_0}(U)$. Since $\varphi$ is continuous, $\varphi^{-n_0}(U)$ is a neighbourhood of $x_0$ as well. Hence there exists $m_0 \in \NN$ such that $\varphi^{m_0}(x_0) \in \varphi^{-n_0}(U)$, i.e.\ $x_0 \in \varphi^{-n_1}(U)$ where $n_1 := n_0+m_0$. Proceeding in this way, we obtain a strictly increasing sequence $(n_k)_{k\in\NN}$ of positive integers such that $x_0 \in \varphi^{-n_k}(U)$ for all $k\in\NN$. Thus $x_0$ belongs to infinitely many $\varphi^{-n}(U)$.
\end{proof}