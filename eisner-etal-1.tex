\subsection{Exercises for chapter 1}

\begin{exercise}
	\label{exer:torus-cesaro}
	For $\lambda \in \mathbb{T}$, prove that
	\begin{equation*}
		\frac{1}{n} \sum_{j=1}^n \lambda^j \text{ converges as } n \to \infty
	\end{equation*}
	and calculate the limit. Is the convergence uniform in $\lambda \in \mathbb{T}$?
\end{exercise}

\begin{proof}[Solution]
	Let $S_n(\lambda) := \frac{1}{n}\sum_{j=1}^n \lambda^j$. If $\lambda=1$, then clearly $S_n(1) = 1$ for all $n\in\NN$. For $\lambda \ne 1$, we have
	\begin{equation*}
		|S_n(\lambda)| = \frac{|\lambda|}{n} \left|\sum_{j=0}^{n-1} \lambda^j \right| = \frac{|\lambda|}{n} \frac{|\lambda^n - 1|}{|\lambda - 1|} \le \frac{2}{n} \frac{|\lambda|}{|\lambda-1|} \longrightarrow 0
	\end{equation*}
	as $n\to\infty$. Hence
	\begin{equation*}
		\lim_{n \to \infty} S_n(\lambda) = \begin{cases}
			0 \qquad \lambda \ne 1 \\
			1 \qquad \lambda = 1,
		\end{cases}
	\end{equation*}
	and since the limit function is discontinuous, the convergence is not uniform.
\end{proof}

\begin{exercise}
	Prove that a convergent sequence $(a_n) \subset \CC$ with limit $a$ is \emph{Ces\`{a}ro convergent} to the same limit, i.e.\ $A_n := \frac{1}{n}\sum_{j=1}^n a_j$ converges to $a$ as $n\to\infty$.
\end{exercise}

\begin{proof}[Solution]
	Let $\varepsilon >0$ be arbitrary. We show that there exists $N\in\NN$ such that $|A_{N+k}-a|$ can be made as small as desired for sufficiently large $k\in\NN$. Observe that
	\begin{align*}
		|A_{N+k}-a| = \left| \frac{1}{N+k}\sum_{j=1}^{N+k}a_j - a\right| &= \left|\frac{1}{N+k}\sum_{j=1}^N (a_j-a) + \frac{1}{N+k}\sum_{j=N+1}^{N+k}(a_j - a)\right| \\
		&\le \frac{M_N}{N+k} + \frac{1}{N+k}\sum_{j=N+1}^{N+k}|a_j-a|,
	\end{align*}
	where $M_N = |\sum_{j=1}^N (a_j-a)|$. Choose $N\in\NN$ such that $|a_j-a| < \varepsilon$ for all $j \ge N$. If $k_0\in\NN$ is chosen large enough to satisfy $M_N \le (N+k_0)\varepsilon$, then
	\begin{equation*}
		|A_{N+k}-a| \le \frac{M_N}{N+k} + \frac{1}{N+k}\sum_{j=N+1}^{N+k}|a_j-a| \le \varepsilon + \frac{k}{N+k} \varepsilon \le 2\varepsilon
	\end{equation*}
	for all $k\ge k_0$. Hence $A_n \to a$ as $n\to\infty$.
\end{proof}

\begin{exercise}
	Give an example of a non-convergent sequence in $\CC$ that is Ces\`{a}ro convergent. Give an example of a bounded sequence that is not Ces\`{a}ro convergent.
\end{exercise}

\begin{proof}[Solution]
	The sequence $a_n = (-1)^n$ is non-convergent, but is Ces\`{a}ro convergent to 0 (this is easily verified directly, or can be seen as a special case of Exercise~\ref{exer:torus-cesaro}).
\end{proof}

\begin{exercise}
	Let $T$ be a $d\times d$ permutation matrix. Prove that
	\begin{equation*}
		\frac{1}{n}\sum_{j=1}^n T^j \text{ converges as } n \to \infty
	\end{equation*}
	and determine the limit.
\end{exercise}

\begin{exercise}
	Let $T$ be a $d\times d$ complex matrix. Prove that $\sup_{n\in\NN} \|T^n\| < \infty$ (i.e.\ the matrix is \emph{power-bounded}) if and only if
	\begin{equation*}
		A_n := \frac{1}{n}\sum_{j=1}^n T^j \text{ converges as } n\to\infty.
	\end{equation*}
	[\emph{Hint}: consider the Jordan normal form of $T$].
\end{exercise}

\subsection{Notes and exercises for chapter 2}
\textbf{Convention}: in this book, a \emph{topological dynamical system} is a pair $(K;\varphi)$, where $K$ is a non-empty compact Hausdorff space, and $\varphi : K \to K$ is continuous.

\begin{definition}
	A subset $A \subseteq K$ is called \emph{invariant} if $\varphi(A) \subseteq A$, \emph{stable} if $\varphi(A) = A$, and \emph{bi-invariant} if $\varphi^{-1}(A) = A$.
\end{definition}

\begin{lemma}[Exercise 2.9]
	Let $(K;\varphi)$ be a topological dynamical system, and let $A \subseteq K$. The following assertions hold:
	\begin{enumerate}[\upshape (i)]
		\item If $A$ is bi-invariant or stable, then $A$ is invariant.
		\item If $A$ is stable and $\varphi$ injective, then $A$ is bi-invariant.
		\item If $A$ is bi-invariant and $\varphi$ surjective, then $A$ is stable.
	\end{enumerate}
\end{lemma}

\begin{proof}
	The proof essentially follows from basic facts about mappings. Let $X, Y$ be non-empty sets and $f:X \to Y$. Then
	\begin{itemize}
		\item For all $A \subseteq X$, $A \subseteq f^{-1}(f(A))$ with equality if and only if $f$ is injective.
		\item For all $B \subseteq Y$, $f(f^{-1}(B)) \subseteq B$ with equality if and only if $f$ is surjective.
	\end{itemize}
	
	(i): Stable trivially implies invariant. If $\varphi^{-1}(A) = A$, then $A \supseteq \varphi(\varphi^{-1}(A)) = \varphi(A)$, so $A$ is invariant.
	
	(ii): Since $\varphi(A)=A$ and $\varphi$ is injective, we obtain $\varphi^{-1}(A) = \varphi^{-1}(\varphi(A)) = A$.
	
	(iii): Since $\varphi^{-1}(A)=A$ and $\varphi$ is surjective, we obtain $\varphi(A) = \varphi(\varphi^{-1}(A)) = A$.
\end{proof}

\begin{lemma}
	Suppose that $(K;\varphi)$ is a topological dynamical system, and $A \subseteq K$ is a non-empty, closed and invariant set. Then there is a non-empty closed set $B \subseteq A$ such that $\varphi(B)=B$, i.e.\ $B$ is stable.
\end{lemma}

\begin{proof}
	Since $K$ is a compact Hausdorff space, $A$ is also compact. Then
	\begin{equation*}
		A \supseteq \varphi(A) \supseteq \varphi^2(A) \supseteq \cdots \supseteq \varphi^n(A)
	\end{equation*}
	holds for all $n\in\NN$, and each of these sets is non-empty and compact, since $\varphi$ is continuous. The collection of closed sets $\{\varphi^n(A)\}_{n\in\NN_0}$ has the finite intersection property, and hence, by the compactness of $K$, the set $B = \bigcap_{n\in\NN} \varphi^n(A)$ is non-empty and compact. Clearly $\varphi(B) \subseteq B$. To prove the reverse inclusion, let $x\in B$ be arbitrary. Then for each $n\in\NN$, we have $x \in \varphi^n(A)$, in other words, $\varphi^{-1}\{x\} \cap \varphi^{n-1}(A) \ne \emptyset$ for all $n\in\NN$. Therefore $\varphi^{-1}\{x\} \cap B \ne \emptyset$, i.e.\ $x\in \varphi(B)$. This completes the proof. 
\end{proof}

\begin{corollary}
	For a topological system $(K;\varphi)$, define
	\begin{equation*}
		K_s := \bigcap_{n\in\NN_0} \varphi^n(K).
	\end{equation*}
	Then $(K_s;\varphi)$ is the unique maximal, hence largest, surjective subsystem of $(K;\varphi)$.
\end{corollary}

\begin{exercise}[Ex.\ 2.11]
	Let $(K;\varphi)$ be a topological system. Show that:
	\begin{enumerate}[(i)]
		\item If $A \subseteq K$ is invariant (resp.\ stable), then $\overline{A}$ is invariant (resp.\ stable).
		\item The intersection of arbitrarily many (bi-)invariant subsets of $K$ is again (bi-)invariant.
		\item If $\Psi:(K;\varphi) \to (L;\psi)$ is a homomorphism of topological systems and $A \subseteq K$ is invariant (resp.\ stable), then so is $\Psi(A) \subseteq L$.
	\end{enumerate}
\end{exercise}

\begin{proof}
	(i): If $A$ is invariant, then $\overline{\varphi(A)} \subseteq \bar{A}$. By continuity of $\varphi$, the set $\varphi^{-1}(\overline{\varphi(A)})$ is closed and clearly contains $A$. Hence $\bar{A} \subseteq \varphi^{-1}(\overline{\varphi(A)})$, which implies $\varphi(\bar{A}) \subseteq \overline{\varphi(A)} \subseteq \bar{A}$.
	
	Now assume $A$ is stable, i.e.\ $\varphi(A)=A$.  Since $K$ is a compact Hausdorff space, every continuous map on $K$ maps closed sets to closed sets. Thus $\varphi(\bar{A})$ is a closed set containing $\varphi(A)=A$, which implies $\overline{\varphi(A)} \subseteq \varphi(\bar{A})$. However, by the above argument, the reverse inclusion also holds, hence $\varphi(\bar{A}) = \bar{A}$.
	
	(ii): This is an immediate consequence of the following general fact: for any mapping $f:K \to K$ and any collection of subsets $\{A_i\}_{i \in I}$ of $K$, it holds that $f^{-1}(\bigcap_{i\in I} A_i) = \bigcap_{i\in I} f^{-1}(A_i)$ and $f(\bigcap_{i\in I} A_i) = \bigcap_{i\in I} f(A_i)$.
	
	(iii): We have $\psi \circ \Psi = \Psi \circ \varphi$. If $A \subseteq K$ is invariant (resp.\ stable), then $\psi(\Psi(A)) = \Psi(\varphi(A)) \subseteq \Psi(A)$, respectively $=\Psi(A)$.
\end{proof}

\subsection{Notes and exercises for chapter 3}

A topological dynamical system is \emph{minimal} if there are no non-trivial, $\varphi$-invariant closed subsets. In other words, if $A \subseteq K$ is closed and invariant, then either $A=\emptyset$ or $A=K$.

\begin{exercise}[Exercise 3.1]
	Prove that a minimal subsystem of an invertible topological system is itself invertible.
\end{exercise}

\begin{proof}
	We show that $\varphi^{-1}(A) = A$. Since $\varphi$ is injective, by Lemma~\ref{lem:invariant-stable}(ii) it suffices to show that $A$ is stable under $\varphi$. By Lemma~\ref{lem:stable-subset}, $A$ contains a non-empty, closed, stable subset $B$. However, if $B$ is a strict subset of $A$, this would contradict the minimality of $A$. Hence $B=A$ is stable.
\end{proof}

\begin{proposition}[Proposition 3.3 / Exercise 3.4]
	For a topological system $(K;\varphi)$, the following are equivalent:
	\begin{enumerate}[\upshape (i)]
		\item $(K;\varphi)$ is minimal.
		\item $\mathrm{orb}_+(x)$ is dense in $K$ for each $x\in K$.
		\item $K = \bigcup_{n\in\NN_0} \varphi^{-n}(U)$ for every non-empty open set $U \subseteq K$.
	\end{enumerate}
	In particular, every minimal system is forward transitive.
\end{proposition}

\begin{proof}
	`(i) $\Rightarrow$ (ii)': Assume that $K$ is minimal, and choose any $x\in K$. Then $A = \overline{\mathrm{orb}_+(x)}$ is a closed, $\varphi$-invariant subset (see Exercise~\ref{exer:2-11}), so $A=K$ by minimality.
	
	`(ii) $\Rightarrow$ (iii)': Let $U \subseteq K$ be non-empty and open. By the continuity of $\varphi$, note that each $\varphi^{-n}(U)$ is open. Fix an arbitrary $x\in K$. By assumption, $\mathrm{orb}_+(x)$ is dense in $K$, and hence has non-empty intersection with some $\varphi^{-n}(U)$.
	
	`(iii) $\Rightarrow$ (i)': Assume for contradiction that there exists a non-empty, closed invariant set $A$ with $A \ne K$. Then the complement $A^c$ is a non-empty open set, and $K = \bigcup_{n\in\NN_0} \varphi^{-n}(A^c)$ by assumption. Note that $A \subseteq \varphi^{-n}(A)$ for all $n\in\NN_0$ (by invariance), and $\emptyset = \varphi^{-n}(A \cap A^c) = \varphi^{-n}(A) \cap \varphi^{-n}(A^c)$. Thus $\varphi^{-n}(A^c)$ does not intersect $A$ for all $n\in\NN_0$, and it follows that $K = \bigcup_{n\in\NN_0} \varphi^{-n}(A^c)$ does not intersect $A$, an absurdity.
	
	\emph{Remark}: note that compactness of $K$ was not used in the proof.
\end{proof}

\begin{exercise}[Exercise 3.6]
	Let $x_0$ be a recurrent point of $(K;\varphi)$. Show that $x_0$ is \emph{infinitely recurrent}, i.e.\ for every neighbourhood $U$ of $x_0$, one has
	\begin{equation*}
		x_0 \in \limsup_n \varphi^{-n}(U) := \bigcap_{n\in\NN} \bigcup_{m\ge n} \varphi^{-m}(U).
	\end{equation*}
\end{exercise}

\begin{proof}
	Fix a neighbourhood $U$ of $x_0$. Since $x_0$ is recurrent, there exists $n_0\in\NN$ such that $\varphi^{n_0}(x_0) \in U$, or equivalently $x_0 \in \varphi^{-n_0}(U)$. Since $\varphi$ is continuous, $\varphi^{-n_0}(U)$ is a neighbourhood of $x_0$ as well. Hence there exists $m_0 \in \NN$ such that $\varphi^{m_0}(x_0) \in \varphi^{-n_0}(U)$, i.e.\ $x_0 \in \varphi^{-n_1}(U)$ where $n_1 := n_0+m_0$. Proceeding in this way, we obtain a strictly increasing sequence $(n_k)_{k\in\NN}$ of positive integers such that $x_0 \in \varphi^{-n_k}(U)$ for all $k\in\NN$. Thus $x_0$ belongs to infinitely many $\varphi^{-n}(U)$.
\end{proof}

\subsection{Notes and exercises for chapter 4}

We combine Theorem 4.8 in [Eisner et al.] with Proposition 2.1.9 in [Meyer-Nieberg] to show that the notions of \emph{algebraic ideal} and \emph{order ideal} coincide in $C(K)$, which is a Banach lattice as well as a (commutative, unital) $C^*$-algebra.

\begin{theorem}[Characterisation of ideals in $C(K)$]
	For every closed subspace $J$ of $C(K)$, the following assertions are equivalent:
	\begin{enumerate}[\upshape (i)]
		\item $J$ is an \emph{algebraic ideal}, i.e.\ if $f \in J$, then $fg \in J$ for all $g \in C(K)$.
		\item $J$ is an \emph{order ideal}, i.e.\ if $f \in J$ and $|g| \le |f|$, then $g \in J$.
		\item There is a closed subset $F \subseteq K$ such that
		\begin{equation}
			J = J_F := \{ f\in C(K) : f = 0 \text{ on } F\}.
		\end{equation}
	\end{enumerate}
\end{theorem}

\begin{proof}
	`(iii) $\Rightarrow$ (i)' :  This is obvious.
	
	`(i) $\Rightarrow$ (ii)': Let $f\in J$, and suppose $g \in C(K)$ satisfies $|g| \le |f|$. For every $\varepsilon >0$, it follows from Urysohn's lemma that there exists $h \in C(K)$ such that $0 \le h \le 1$, $h=0$ on $[|f| \le \varepsilon/2]$ and $h=1$ on $[|f| \ge \varepsilon]$. In particular, the support of $h$ is contained in $[|f| \ge \varepsilon/2]$. Therefore the function
	\begin{equation*}
		g_0 := \frac{gh}{f}
	\end{equation*}
	is well-defined, and $gh = g_0f \in J$. Consequently
	\begin{align*}
		|g(x)-g_0(x) f(x)| &= |g(x)[h(x)+(1-h(x))]-g_0(x) f(x)| \\
		&= |g(x)(1-h(x))| \le |f(x)(1-h(x))| \le \varepsilon
	\end{align*}
	for all $x\in K$. This shows that $g$ may be approximated arbitrarily well by elements in $J$, and because $J$ is closed, we conclude that $g\in J$.
	
	`(ii) $\Rightarrow$ (iii)': Define the set
	\begin{equation*}
		F := \bigcap_{f\in J} [f=0] = \{x\in K : f(x)=0 \text{ for all } f\in J\}.
	\end{equation*}
	Clearly $F$ is closed, and $J \subseteq J_F$. To prove that $J_F \subseteq J$, fix $f\in J_F$ and $\varepsilon >0$, and define $F_\varepsilon := [|f| \ge \varepsilon]$. Since $f$ is continuous and vanishes on $F$, $F_\varepsilon$ is a closed subset of $K \setminus F$ (and compact in $K$). Hence for each $x\in F_\varepsilon$, we can find a function $f_x \in J$ such that $f_x(x) \ne 0$. Upon multiplying by $\overline{f_x}$ and a positive constant, we may assume $f_x \ge 0$ and $f_x(x)>1$. Then the collection of open sets $([f_x > 1])_{x\in F_\varepsilon}$ covers $F_\varepsilon$. By compactness of $F_\varepsilon$, there are consequently $0 \le f_1, \ldots, f_k \in J$ such that
	\begin{equation*}
		F_\varepsilon \subseteq \bigcup_{j=1}^k [f_j > 1].
	\end{equation*}
	Let $g := f_1+\ldots +f_k \in J$. Then $g \ge 0$, and $F_\varepsilon \subseteq [g \ge 1]$. Define
	\begin{equation*}
		g_n := \frac{nf}{1+ng}g \in J \qquad (n\in\NN).
	\end{equation*}
	Noting that $g \ge 1$ on $F_\varepsilon$ and $|f|<\varepsilon$ on $K \setminus F_\varepsilon$, we find that
	\begin{equation*}
		|g_n - f| = \frac{|f|}{1+ng} \le \max\left\{ \varepsilon, \frac{\|f\|_\infty}{1+n} \right\} \le \varepsilon
	\end{equation*}
	for all sufficiently large $n\in\NN$. Since $J$ is closed, we conclude that $f\in J$.
\end{proof}

The above Theorem shows that closed ideals of $C(K)$ are in one-to-one correspondence with closed subsets of $K$. If $V \subseteq W$ are closed subsets of $K$, then $J_V \supseteq J_W$. This allows us to identify maximal ideals easily. Recall that the Dirac functional $\delta_x$ at $x$ is defined by evaluation, $\braket{\delta_x, f} := f(x)$.
\begin{corollary}[Maximal ideals in $C(K)$]
	An ideal $J$ in $C(K)$ is maximal if and only if $J = J_{\{x\}}$ for some $x\in K$; equivalently, $J = \ker(\delta_x)$, where $\delta_x$ is the Dirac functional at $x$.
\end{corollary}

\begin{proof}
	It is clear that any ideal of the form $J_{\{x\}}$ must be maximal. Indeed, if $I$ is an ideal strictly larger than $J_{\{x\}}$, then $I$ contains the constant function $\mathbf{1}$, and thus $I = C(K)$.
	
	Conversely, let $J$ be a maximal ideal. Since the closure of an ideal is again an ideal, either $\overline{J} = J$ (so $J$ is closed), or $\overline{J} = C(K)$ (i.e.\ $J$ is dense in $C(K)$). In the latter case, there exists $f \in J$ such that $\|f-\mathbf{1}\| < 1$. Then $f = \mathbf{1}-(\mathbf{1}-f)$ is bounded away from 0, and thus $1/f \in C(K)$. However this implies $f \cdot (1/f) = \mathbf{1} \in J$, hence $J = C(K)$, contradicting the maximality of $J$. Thus $J$ is a closed ideal, and Theorem~\ref{thm:CK-ideal}(iii) and the remarks preceding the corollary show that $J = J_{\{x\}}$ for some $x\in K$.
\end{proof}

For the next result, we will use the following elementary fact from functional analysis: if $X,Y$ are Banach spaces and $T:X\to Y$ is a bounded linear operator, then $T$ is surjective if and only if $T':Y'\to X'$ is injective. Indeed, suppose that $T$ is surjective, and take $y'\in \ker T' \subset Y'$. Then
\begin{equation*}
	0 = \braket{T'y',x} = \braket{y',Tx} = \braket{y',y}
\end{equation*}
for all $y\in Y$. Thus $y'=0$, showing that $T'$ is injective. Conversely, if $T$ is not surjective, then $T(X)$ is a proper closed subspace of $Y$. By the Hahn-Banach theorem, there exists a non-zero functional $y'$ such that $\braket{y',Tx} = \braket{T'y',x}=0$ for all $x\in X$. In other words, $T'y'=0$, so $T'$ is not injective.

\begin{lemma}[Lemma 4.14 / Exercise 4.1]
	Let $K, L$ be compact Hausdorff spaces. Let $\varphi:L\to K$ be a continuous map, and $T_\varphi$ its associated Koopman operator $T_\varphi: C(K)\to C(L)$ defined by $T_\varphi f := f \circ \varphi$. Then
	\begin{enumerate}[\upshape (i)]
		\item $\varphi$ is surjective if and only if $T_\varphi$ is injective. In this case, $T_\varphi$ is an isometry.
		\item $\varphi$ is injective if and only if $T_\varphi$ is surjective.
	\end{enumerate}
\end{lemma}

\begin{proof}
	Let us write $T := T_\varphi$.
	
	(i): Assume that $\varphi$ is surjective and $Tf = 0$. Then $(f\circ\varphi)(x)=0$ for all $x\in L$. However, for all $y\in K$, there exists $x\in L$ with $\varphi(x)=y$. Hence $f(y)=0$ for all $y\in K$, which shows that $\ker T = \{0\}$. Hence $T$ is injective.
	
	Conversely, if $\varphi$ is not surjective, then $\varphi(L)$ is a proper closed subset of $K$. For each $y\in K \setminus \varphi(L)$, Urysohn's lemma gives a continuous function $f$ such that $0 \le f \le 1$, $f=0$ on $\varphi(L)$ and $f(y)=1$. For this $f$, we have $(Tf)(x) = (f\circ\varphi)(x) = 0$ for all $x\in L$, i.e.\ $f \in \ker T$ and $f \not\equiv 0$. Thus $T$ is not injective.
	
	For every $f \in C(K)$, it is clear that $\|Tf\|_\infty \le \|f\|_\infty$. Since $f$ is continuous on a compact set, the maximum of $|f|$ is attained at some $y_0\in K$, and by surjectivity of $\varphi$, there exists $x_0\in L$ with $\varphi(x_0)=y_0$. Hence $|(Tf)(x_0)| = |(f\circ\varphi)(x_0)| = |f(y_0)| = \|f\|_\infty$, which proves that $\|Tf\|_\infty = \|f\|_\infty$, i.e.\ $T$ is an isometry.
	
	(ii): (ii): This is a `dual' statement to (i). Suppose $T$ is surjective, so $T'$ is injective by the remarks preceding the lemma. Suppose that $\varphi(x)=\varphi(y)$ for some $x,y \in L$. Then $(f\circ\varphi)(x)=(f\circ\varphi)(y)$ for all $f\in C(K)$, which can be rewritten as
	\begin{equation*}
		\braket{T'\delta_x,f} = \braket{T'\delta_y,f} \qquad \text{for all } f \in C(K).
	\end{equation*}
	Thus $\delta_x = \delta_y$ by the injectivity of $T'$, or equivalently $x=y$. Hence $\varphi$ is injective.
	
	Conversely, suppose $\varphi$ is injective, and let $\mu\in\ker T' \subset C(L)'$. If $x\ne y$, then $\varphi(x)\ne \varphi(y)$ by injectivity. By Urysohn's lemma, there exists $f\in C(K)$ such that $0\le f\le 1$, $(Tf)(y)=(f\circ\varphi)(y)=0$ and $(Tf)(x)=(f\circ\varphi)(x)=1$. However $\braket{T'\mu, f} = 0$, which is possible only if $\mu=0$. Hence $T'$ is injective, or equivalently $T$ is surjective.
\end{proof}

\begin{remark}
	It is worth looking more closely at the implication $\delta_x = \delta_y \Rightarrow x=y$. If $x\ne y$, then by Urysohn's lemma we can separate $x$ from $y$ by a continuous function $f$. For this $f$, we clearly have $\braket{\delta_x,f} \ne \braket{\delta_y,f}$, so that $\delta_x$ and $\delta_y$ are distinct linear functionals.
\end{remark}

\begin{remark}
	By associating each compact Hausdorff space $K$ with $C(K)$ together with the mapping $\varphi \mapsto T_\varphi$, we obtain a \emph{contravariant functor} from the category of compact Hausdorff spaces $\mathbf{CHaus}$ to the category of $C^*$-algebras. Let us verify this in detail:
	\begin{itemize}
		\item If $\varphi:L \to L$ is the identity map on $L$, then $(T_\varphi f)(x) = f(x)$ for all $x\in L$, hence $T_\varphi$ is the identity map $f \mapsto f$ on $C(L)$.
		\item If $L,K \in \mathbf{CHaus}$ and $\varphi:L \to K$ is a continuous map, then $T_\varphi : C(K) \to C(L)$ is a $^*$-homomorphism. Indeed, it is clearly an algebra homomorphism, since if $f,g \in C(K)$, then $T_\varphi(fg) = (fg)\circ \varphi = (f\circ\varphi)(g\circ\varphi) = T_\varphi f \cdot T_\varphi g$. The involution on $C(K)$ is given by complex conjugation, and we note that $(T_{\varphi}\bar{f})(x) = (\bar{f}\circ\varphi)(x) = \overline{f(\varphi(x))}$ for all $x\in K$, which shows that $T_\varphi \bar{f} = \overline{T_\varphi f}$. Thus $T_\varphi$ is a $^*$-homomorphism as claimed.
		\item If $K,L,M \in \mathbf{CHaus}$ and $\varphi:K \to L$ and $\psi:L \to M$ are continuous, then $T_{\psi\circ\varphi}:C(M) \to C(K)$ satisfies
		\begin{equation*}
			T_{\psi \circ \varphi}f = f\circ\psi\circ\varphi = T_\varphi(f\circ\psi) = (T_\varphi \circ T_\psi)f
		\end{equation*}
		for all $f\in C(M)$.
	\end{itemize}
	Thus we have verified all the properties of a contravariant functor.
\end{remark}