\subsection{Some important examples}

\begin{example}[$C(\Omega)$ spaces]
	Let $\Omega \subseteq \RR^N$ be a non-empty open set. Then $\Omega$ is the union of countably many compact sets $K_n$ which may be chosen such that $K_n \subseteq \text{int}(K_{n+1})$ for all $n \ge 1$. We define the space $C(\Omega)$ of complex-valued functions on $\Omega$, topologised by the separating family of seminorms
	\begin{equation*}
		p_n(f) := \sup_{K_n} |f(x)|.
	\end{equation*}
	Since $p_1 \le p_2 \le \cdots$, it follows that the sets
	\begin{equation*}
		V_n = \left\{ f \in C(\Omega) : p_n(f) < \frac{1}{n} \right\} \qquad (n=1,2,3,\ldots)
	\end{equation*}
	form a convex local base for $C(\Omega)$. By Remark 1.38(c), the topology generated by this local base is compatible with the metric
	\begin{equation*}
		d(f,g) := \max_{n \in \NN} \frac{2^{-n} p_n(f-g)}{1+p_n(f-g)}.
	\end{equation*}
	If $(f_i)$ is a Cauchy sequence relative to this metric, then clearly $p_n(f_i - f_j) \to 0$ for every $n \in \NN$ as $i,j \to \infty$. By elementary real analysis, $(f_i)$ converges uniformly on $K_n$ to some $f \in C(\Omega)$. It is then easy to check that $d(f, f_i) \to 0$ as $i \to \infty$. Hence $d$ is complete, and $C(\Omega)$ is a \emph{Fr\'{e}chet space} (locally convex tvs with a complete, translation invariant metric).
	
	By Theorem 1.37(b), a subset $E \subset C(\Omega)$ is bounded if and only if every seminorm $p_n$ is bounded on $E$, i.e.\ there exist numbers $M_n < \infty$ such that $p_n(f) \le M_n$ for all $f \in E$. Explicitly, $|f(x)| \le M_n$ if $x \in K_n$, for all $f \in E$. However, note that each set $V_n$ contains an $f$ for which $p_{n+1}(f)$ can be made as large as we like. Hence no $V_n$ is bounded, and by Theorem 1.39, it follows that $C(\Omega)$ is not normable.
\end{example}

\begin{example}
	Let $\Omega$ now be an open subset of the complex plane, and define $C(\Omega)$ as above. Consider the subspace $H(\Omega)$ of $C(\Omega)$ consisting of holomorphic functions on $\Omega$. Recall that sequences of holomorphic functions that converge uniformly on compact sets have holomorphic limits (this follows easily by applying Morera's theorem). Thus $H(\Omega)$ is a closed subspace of $C(\Omega)$, and is a Fr\'{e}chet space in its own right.
	
	(Some comment about Montel's theorem and Heine-Borel property)
\end{example}

\begin{example}[Test functions and distributions]
	(complete later)
\end{example}

\subsection{Rudin Chapter 1 Exercises}

\begin{exercise}
	Let $X$ be a vector space. All sets mentioned below are understood to be subsets of $X$. Prove the following statements from the axioms as given in Section 1.4.
	\begin{enumerate}[(a)]
		\item $2A \subseteq A+A$, and it may happen that $2A \ne A+A$.
		\item $A$ is convex if and only if $(s+t)A = sA + tA$ for all $s,t>0$.
		\item Every union and intersection of balanced sets is balanced.
		\item Every intersection of convex sets is convex.
		\item If $\Gamma$ is a collection of convex sets that is totally ordered by inclusion, then the union of all members of $\Gamma$ is convex.
		\item If $A, B$ are convex, so is $A+B$.
		\item If $A, B$ are balanced, so is $A+B$.
		\item Show that (d), (e), (f) hold with subspaces in place of convex sets.
	\end{enumerate}
\end{exercise}

\begin{proof}[Solution]
	\begin{enumerate}[(a)]
		\item It follows from the definition that
		\begin{equation*}
			2A = \{ 2a : a \in A \} = \{ a+a : a \in A \} \subseteq A+A.
		\end{equation*}
		Consider $X = \RR$, and let $A = \{0, 1\}$. Then $A+A = \{0,1,2 \}$ but $2A = \{0,2\}$.
		
		\item Firstly, note that for any non-empty subset $A$, it holds that
		\begin{equation*}
			(s+t)A = \{ (s+t)a : a \in A \} = \{ sa + ta : a \in A \} \subseteq sA + tA
		\end{equation*}
		for all $s,t>0$.
		
		`$\Rightarrow$': Assume $C$ is convex. Then
		\begin{equation*}
			\frac{s}{s+t}C + \frac{t}{s+t}C \subseteq C \qquad \text{for all } s,t>0.
		\end{equation*}
		This is equivalent to $sC + tC \subseteq (s+t)C$. Combining with the above remark, it follows that $sC + tC = (s+t)C$.
		
		`$\Leftarrow$': If $(s+t)C = sC + tC$ for all $s,t>0$, then we may choose $s,t$ such that $s+t=1$. Hence $sC + tC = C$ which proves that $C$ is convex.
		
		\item Let $\{ B_\alpha \}_{\alpha \in I}$ be an arbitrary collection of balanced sets. Write $U = \bigcup_\alpha B_\alpha$ and $V = \bigcap_\alpha B_\alpha$. First observe that for any scalar $\lambda$, it holds that
		\begin{equation*}
			\lambda U = \{ \lambda x : x \in B_\alpha \text{ for some } \alpha \in I \} = \bigcup_\alpha \lambda B_\alpha
		\end{equation*}
		and similarly for the intersection (where `for some' is replaced by `for all'). Now suppose $|\lambda| \le 1$. Then $\lambda U = \bigcup_\alpha \lambda B_\alpha \subseteq \bigcup_\alpha B_\alpha = U$, hence $U$ is balanced. Similarly, $\lambda V = \bigcap_\alpha \lambda B_\alpha \subseteq \bigcap_\alpha B_\alpha = V$, so $V$ is balanced.
		
		\item Let $\{ C_\alpha \}_{\alpha\in I}$ be an arbitrary collection of convex sets, and consider $C = \bigcap_\alpha C_\alpha$. Let $x,y \in C$ and $t \in (0,1)$ be arbitrary. Since each $C_\alpha$ is convex, it follows that $(1-t)x + ty \in C_\alpha$ for every $\alpha$; whence $(1-t)x + ty \in \bigcap_\alpha C_\alpha = C$. Hence $C$ is convex.
		
		\item Let $\Gamma = \{C_\alpha\}_{\alpha\in J}$ be a collection of convex sets totally ordered by inclusion, and set $C = \bigcup_\alpha C_\alpha$. Let $x,y \in C$ and $t\in (0,1)$ be arbitrary. Then $x \in C_{\alpha_1}, y \in C_{\alpha_2}$ for some indices $\alpha_1, \alpha_2 \in J$. However, since $\Gamma$ is totally ordered, without loss of generality we may assume $C_{\alpha_1} \subseteq C_{\alpha_2}$. By the convexity of $C_{\alpha_2}$, it follows that $(1-t)x + ty \in C_{\alpha_2} \subseteq C$, hence $C$ is convex.
		
		\item Let $x,y \in A+B$ and $t \in (0,1)$ be arbitrary. Then there exist $a_i \in A, b_i \in B, i\in\{1,2\}$, such that $x=a_1+b_1$ and $y=a_2+b_2$. Then
		\begin{align*}
			(1-t)x+ty &= (1-t)(a_1+b_1) + t(a_2+b_2) \\
			&= [(1-t)a_1+ta_2] + [(1-t)b_1+tb_2] \in A+B
		\end{align*}
		by the convexity of $A$ and $B$. Hence $A+B$ is convex.
		
		\item Suppose that $|\lambda| \le 1$. Then
		\begin{align*}
			\lambda(A+B) &= \{ \lambda(x+y) : x\in A, y\in B \} \\
			&= \{ \lambda x + \lambda y : x\in A, y\in B\} \\
			&= \lambda A + \lambda B \subseteq A + B
		\end{align*}
		since $A, B$ are balanced. This shows that $A+B$ is balanced.
		
		\item Obvious :)
	\end{enumerate}
\end{proof}

\begin{remark}
	Let $X, Y$ be topological vector spaces over the same scalar field, and suppose $T : X \to Y$ is a linear map. Here are some useful facts to keep in mind:
	\begin{enumerate}[(i)]
		\item If $A$ is a subspace, respectively a convex set or a balanced set in $X$, then $T(A)$ is a subspace, resp.\ convex, or balanced in $Y$.
		\item If $B$ is a subspace, resp.\ convex or balanced in $Y$, then $T^{-1}(B)$ is a subspace, resp.\ convex or balanced in $X$.
	\end{enumerate}
\end{remark}

\begin{exercise}
	The \textbf{convex hull} of a set $A$ in a vector space $X$ is the set of all \textbf{convex combinations} of members of $A$, that is, the set of all sums
	\begin{equation*}
		t_1 x_1 + \ldots + t_n x_n
	\end{equation*}
	where $t_i \ge 0$ with $\sum t_i = 1$, $x_i \in A$ and $n \in \NN$ are arbitrary. Prove that the convex hull of A, denoted by $\co A$, is convex, and that it is the intersection of all convex sets containing $A$.
\end{exercise}

\begin{proof}
	Let $x, y \in \co A$. Then $x, y$ can be expressed as convex combinations
	\begin{equation*}
		x = \sum_{i=1}^n t_i x_i, \qquad y = \sum_{j=1}^m s_j y_j
	\end{equation*}
	where $\sum t_i = \sum s_j = 1, t_i \ge 0, s_j \ge 0$ and $x_i, y_j \in A$ for all $1\le i \le n$ and $1\le j \le m$. Now let $p \in (0,1)$ be arbitrary. We can relabel the scalars $t_i, s_j$ as $\{\alpha_k\}_{1\le k \le n+m}$, and similarly we relabel the vectors $x_i, y_j$ as $\{z_k\}_{1\le k \le n+m}$. Then
	\begin{equation*}
		(1-p)x + py = \sum_{i=1}^n (1-p)t_i x_i + \sum_{j=1}^m ps_j y_j = \sum_{k=1}^{n+m} \alpha_k z_k,
	\end{equation*}
	where we observe that
	\begin{equation*}
		\sum_{k=1}^{n+m}\alpha_k = (1-p)\sum_{i=1}^n t_i + p\sum_{j=1}^m s_j = (1-p) + p = 1.
	\end{equation*}
	It follows that $(1-p)x + py \in \co A$.
	
	If $C$ is a convex set containing $A$, then $C$ contains all convex combinations of elements in $A$. In other words, $\co A \subseteq C$. It follows that $\co A \subseteq \bigcap C$ where the intersection is taken over all convex sets containing $A$. This intersection is clearly the smallest convex set containing $A$. Since $\co A$ is convex and contains $A$, the reverse inclusion $\bigcap C \subseteq \co A$ follows.
\end{proof}

\begin{exercise}
	Let $X$ be a tvs. All sets mentioned below are understood to be subsets of $X$. Prove the following statements.
	\begin{enumerate}[(a)]
		\item The convex hull of every open set is open.
		\item If $X$ is locally convex, then the convex hull of every bounded set is bounded. (Note that this is false without local convexity, see Rudin 1.47).
		\item If $A$ and $B$ are bounded, so is $A+B$.
		\item If $A$ and $B$ are compact, so is $A+B$.
		\item If $A$ is compact and $B$ is closed, then $A+B$ is closed.
		\item The sum of two closed sets need not be closed (in particular, the inclusion $\overline{A}+\overline{B} \subseteq \overline{A+B}$ can be strict).
	\end{enumerate}
\end{exercise}

\begin{proof} 
	\begin{enumerate}[(a)]
		\item Let $U$ be an open set in $X$, and fix $x \in \co U$. Then $x$ can be written as $x = \sum_{i=1}^n t_i x_i$ for appropriate $x_i \in U$ and $t_i \ge 0, \sum t_i = 1$. Since $U$ is open, to each $x_i$ corresponds an open neighbourhood $U_i$ such that $x_i \in U_i \subset U$. It follows that
		\begin{equation*}
			x \in \sum_{i=1}^n t_i U_i \subseteq \co U
		\end{equation*}
		and moreover $\sum_{i=1}^n t_i U_i$ is open. Thus every $x \in \co U$ is an interior point, which shows that $\co U$ is open.
		
		\item Let $A$ be a bounded set, and fix an arbitrary neighbourhood $V$ of 0. Since $X$ is locally convex and $A$ is bounded, Theorem 1.14 shows that there exist a convex neighbourhood $W$ of 0 and $t_0>0$ such that $W \subseteq V$ and $A \subseteq tW \subseteq tV$ for all $t>t_0$. Since $tW$ is convex, consequently
		\begin{equation*}
			\co A \subseteq \co(tW) = tW \subseteq tV
		\end{equation*}
		for all $t>t_0$, which shows that $\co A$ is bounded.
		
		\item Let $V$ be an arbitrary neighbourhood of 0 in $X$. Since $0+0=0$ and addition is continuous, there exists a neighbourhood $W$ of 0 such that $W+W \subseteq V$. Since $A, B$ are bounded, there exists $t_0>0$ such that $A, B$ are both contained in $tW$ for all $t>t_0$. Therefore $A+B \subseteq t(W+W) \subseteq tV$ for all $t>t_0$, so $A+B$ is bounded.
		
		\item If $A, B$ are compact in $X$, then $A \times B$ is compact in $X \times X$. Since addition is a continuous function from $X \times X$ into $X$, it follows that $A+B$ is compact in $X$.
		
		\item
		
		\item
	\end{enumerate}
\end{proof}

\begin{exercise}
	Let $B = \{ (z_1,z_2) \in \CC^2 : |z_1|\le |z_2| \}$. Show that $B$ is balanced but $\text{int}(B)$ is not. [Compare with Theorem 1.13(e)].
\end{exercise}

\begin{proof}[Solution]
	Firstly we show that $B$ is balanced. Since $0 \in B$, the assertion is trivial for $\lambda=0$. Hence let $\lambda \in \CC$ such that $0 < |\lambda| \le 1$. If $z = (z_1, z_2) \in B$, then obviously $|\lambda z_1| = |\lambda||z_1| \le |\lambda| |z_2| = |\lambda z_2|$, so $\lambda z \in B$. This shows that $B$ is balanced. However, $\text{int}(B) = \{ (z_1,z_2) \in \CC^2 : 0<|z_1|<|z_2|\}$ which is not balanced, since it does not contain 0.
\end{proof}

\begin{exercise}
	Consider the definition of `bounded set' in Section 1.6. Would the content of this definition be altered if it were required merely that to every neighbourhood $V$ of 0 corresponds \emph{some} $t>0$ such that $E \subseteq tV$?
\end{exercise}

\begin{proof}[Solution]
	No. Let us say that a set $E \subseteq X$ satisfies property $(P)$ if for every neighbourhood $V$ of 0, there exists some $t>0$ such that $E \subseteq tV$. We show that all sets with property $(P)$ are bounded. Let $E$ satisfy property $(P)$ and choose an arbitrary neighbourhood $V$ of 0. By Theorem 1.14(a), there is a balanced neighbourhood $W \subseteq V$ of 0. By property $(P)$, there is $t>0$ such that $E \subseteq tW \subseteq tV$. If $s>t$, then write $t = \lambda s$ for some $\lambda \in (0,1)$. Since $sW$ is balanced, we have $tW = \lambda sW \subseteq sW$. Hence $E$ is contained in $sW$, and in particular contained in $sV$ for all $s>t$. This proves that $E$ is bounded.
\end{proof}

\begin{exercise}
	Prove that a set $E$ in a tvs is bounded if and only if every countable subset of $E$ is bounded.
\end{exercise}

\begin{proof}[Solution]
	The `only if' direction is trivial. Thus let $E$ be a non-empty subset such that every countable subset of $E$ is bounded. Take an arbitrary sequence $(x_n) \subset E$ and a sequence of scalars $(\alpha_n)$ such that $\alpha_n \to 0$ as $n \to \infty$. We will show that $\alpha_n x_n \to 0$, and thus the boundedness of $E$ follows by Theorem 1.30. Let $V$ be an arbitrary neighbourhood of 0. By Theorem 1.14(a), without loss of generality, we may assume that $V$ is balanced. Then we may assume that $|\alpha_n| >0$ for all $n$. Since $|\alpha_n|^{-1} \to \infty$, by the boundedness of $(x_n)$ there exists some $m \in \NN$ such that $(x_n) \subset |\alpha_m|^{-1}V$. Then for sufficiently large $n$, we have $|\alpha_n||\alpha_m|^{-1} \le 1$, so that
	\begin{equation*}
		\alpha_n x_n \in \alpha_n |\alpha_m|^{-1}V \subseteq V
	\end{equation*}
	for sufficiently large $n$. Hence $\alpha_n x_n \to 0$ as claimed.
\end{proof}

\begin{exercise}[Ex.\ 1.8]
	\begin{enumerate}[(a)]
		\item Suppose $\mathscr{P}$ is a separating family of seminorms on a vector space $X$. Let $\mathscr{Q}$ be the smallest family of seminorms on $X$ containing $\mathscr{P}$ that is closed under max, i.e.\ if $p_1, p_2 \in \mathscr{Q}$ and $p := \max(p_1, p_2)$, then $p \in \mathscr{Q}$. If the construction of Theorem 1.37 is applied to $\mathscr{P}$ and $\mathscr{Q}$, show that the two resulting topologies coincide. (The main difference is that $\mathscr{Q}$ leads directly to a base, rather than a sub-base).
		
		\item Let $\mathscr{Q}$ be as in (a), and $\Lambda$ a linear functional on $X$. Show that $\Lambda$ is continuous if and only if there exists a $p \in \mathscr{Q}$ such that $|\Lambda x| \le M p(x)$ for all $x \in X$ for some constant $M < \infty$.
	\end{enumerate}
\end{exercise}

\begin{proof}[Solution]
	As in Theorem 1.37, for each $p \in \mathscr{P}$ or $\mathscr{Q}$ and every $n\in\NN$, we set
	\begin{equation*}
		V(p,n) := \{x \in X : p(x) < 1/n\}.
	\end{equation*}
	(a): It is clear that $\mathscr{Q}$ generates a finer topology than $\mathscr{P}$, since all the sets $V(p,n)$ as $p$ runs over $\mathscr{P}$ and $n$ runs over $\NN$ already belong to the topology generated by $\mathscr{Q}$. Now if $p \in \mathscr{Q}$ and is not an element of $\mathscr{P}$, there exist finitely many $p_1, \ldots, p_k \in \mathscr{P}$ such that $p = \max(p_1, \ldots, p_k)$. If $x \in \bigcap_{i=1}^k V(p_i, n)$, i.e.\ $p_i(x) < 1/n$ for all $i=1,\ldots,k$, then $\max_{1\le i\le k}p_i(x) < 1/n$, which means exactly that $x \in V(p,n)$. Hence
	\begin{equation*}
		\bigcap_{i=1}^k V(p_i,n) \subseteq V(p,n),
	\end{equation*}
	and the intersection on the left hand side is an open set in the topology generated by $\mathscr{P}$. Thus the two topologies coincide (and note moreover that the inclusion above is actually an equality).
	
	(b): Suppose that there exists $p \in \mathscr{Q}$ such that $|\Lambda x| \le Mp(x)$ for all $x \in X$ and some constant $M<\infty$. If $(x_n) \subset X$ is a sequence satisfying $x_n \to 0$, then the continuity of $p$ and the inequality implies that $\Lambda x_n \to 0$ as $n \to \infty$. Thus $\Lambda$ is continuous by Theorem 1.32.
	
	Conversely, assume $\Lambda$ is continuous. Then there exists some $p\in\mathscr{P}$ and $n\in\NN$ such that $\Lambda(V(p,n)) \subset (-1,1)$, which implies $-1 \le \Lambda x \le 1$ for all $x \in X$ such that $p(x)<1/n$. Fix $\varepsilon > 0$. If $x\in X$ is arbitrary, then $x[(n+\varepsilon)p(x)]^{-1}$ belongs to $V(p,n)$, and thus
	\begin{equation*}
		-1 \le \Lambda\left(\frac{x}{(n+\varepsilon)p(x)}\right) \le 1
	\end{equation*}
	which is evidently equivalent to
	\begin{equation*}
		|\Lambda x| \le (n+\varepsilon)p(x)
	\end{equation*}
	for all $x\in X$.
\end{proof}