This section consists of notes and exercises from Chapter 1 of Murphy's \emph{$C^*$ algebras and operator theory}.

Notation: for an abelian algebra $A$, $\Omega(A)$ is the set of all \emph{characters} of $A$, that is, the non-zero homomorphisms $\tau:A\to\CC$.
\begin{exercise}
	Let $A$ be a unital abelian Banach algebra.
	\begin{enumerate}[(a)]
		\item Show that $\sigma(a+b)\subseteq\sigma(a)+\sigma(b)$ and $\sigma(ab)\subseteq\sigma(a)\sigma(b)$. Show by example that this is not true for all Banach algebras.
		\item Show that if $A$ contains an \emph{idempotent} $p=p^2$ other than $0$ and $1$, then $\Omega(A)$ is disconnected.
		\item Suppose that $A$ is generated by $\{a_1, \ldots, a_n\}$. Show that $\Omega(A)$ is homeomorphic to a compact subset of $\CC^n$. More specifically, set
		\begin{equation*}
			\sigma(a_1,\ldots,a_n) := \{ (\tau(a_1),\ldots,\tau(a_n)) : \tau\in\Omega(A) \},
		\end{equation*}
		and show that the canonical map from $\Omega(A)$ to $\sigma(a_1,\ldots,a_n)$ is a homeomorphism.
	\end{enumerate}
\end{exercise}

\begin{proof}[Solution]
	(a): Both inclusions follow directly from the Gelfand representation theorem. Indeed, each element $a$ is represented by a continuous function $\hat{a}\in C_0(\Omega(A))$ such that $\rg(\hat{a})=\sigma(a)$. Since the inclusions $\rg(\hat{a}+\hat{b})\subseteq\rg(\hat{a})+\rg(\hat{b})$ and $\rg(\hat{a}\hat{b})\subseteq\rg(\hat{a})\rg(\hat{b})$ clearly hold for continuous, we obtain the corresponding inclusions of the spectra.
	
	\textcolor{red}{Insert matrix example}
	
	(b): Recall the following characterisation of connectedness: a topological space $X$ is connected if and only if there does not exist a continuous, surjective map into the two-point space $\{0,1\}$. Now let $p\in A$ be idempotent. Since $p(1-p)=0$, observe that
	\begin{equation*}
		0=\tau(p(1-p))=\tau(p)\tau(1-p)=\tau(p)(1-\tau(p)).
	\end{equation*}
	Hence we may define a natural map $\psi : \Omega(A)\to\{0,1\}$ by
	\begin{equation*}
		\Omega(A) \ni \tau \mapsto \psi(\tau) := \tau(p)\in\{0,1\}.
	\end{equation*}
	Since $p\ne 0$ and $p\ne 1$, the map $\psi$ is surjective. Now observe that $\psi^{-1}(0) = \{\tau\in\Omega(A):\tau(p)=0\}$ and $\psi^{-1}(1)=\{\tau\in\Omega(A):\tau(p)=1\}$ are closed sets in the weak$^*$ topology on $\Omega(A)$. This shows that $\psi$ is continuous, and thus $\Omega(A)$ is disconnected.
	
	(c):
\end{proof}

\begin{exercise}
	Let $A$ be a unital Banach algebra.
	\begin{enumerate}[(a)]
		\item If $a\in A$ is invertible, show that $\sigma(a^{-1})=\{\lambda^{-1}:\lambda\in\sigma(a)\}$.
		\item For any element $a\in A$, show that $\mathrm{r}(a^n)=\mathrm{r}(a)^n$ for all $n\in\NN$.
		\item If $A$ is abelian, show that the Gelfand transform is isometric if and only if $\|a^2\|=\|a\|^2$ for all $a\in A$.
	\end{enumerate}
\end{exercise}

\begin{exercise}
	Let $A$ be a Banach algebra. Show that the spectral radius function $\mathrm{r}:A\to\RR$ is upper semi-continuous. (In general it is not continuous, see Halmos Problem 104).
\end{exercise}