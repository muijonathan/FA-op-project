\documentclass[11pt]{article}

%% Language and font encodings
\usepackage[english]{babel}
\usepackage[T1]{fontenc}
\usepackage{newtxtext}

%% Sets page size and margins
\usepackage[a4paper,top=3cm,bottom=3cm,left=3cm,right=3cm,marginparwidth=2cm]{geometry}

%% Useful packages
\usepackage{amsmath, amsthm, amssymb, mathrsfs}
\usepackage{enumerate, braket}
\usepackage[colorlinks=true]{hyperref}
\hypersetup{urlcolor=blue, linkcolor=blue, citecolor=red}

\numberwithin{equation}{section}

\theoremstyle{plain}
\newtheorem{theorem}{Theorem}[section]
\newtheorem{proposition}[theorem]{Proposition}
\newtheorem{lemma}[theorem]{Lemma}
\newtheorem{corollary}[theorem]{Corollary}

\theoremstyle{definition}
\newtheorem{exercise}{Exercise}[section]
\newtheorem{definition}{Definition}[section]
\newtheorem{example}{Example}[section]
%\theoremstyle{remark}
\newtheorem{remark}{Remark}[section]
\newtheorem{assumption}{Assumption}

%%% Shortcuts
\newcommand{\NN}{\mathbb{N}}
\newcommand{\ZZ}{\mathbb{Z}}
\newcommand{\QQ}{\mathbb{Q}}
\newcommand{\RR}{\mathbb{R}}
\newcommand{\CC}{\mathbb{C}}

\DeclareMathOperator{\rg}{\mathrm{rg}}
\DeclareMathOperator{\dist}{\mathrm{dist}}
\renewcommand{\Re}{\mathrm{Re}}
\renewcommand{\Im}{\mathrm{Im}}
\newcommand{\mfa}{\mathfrak{a}}

\title{Research Notes}
\author{Jonathan Mui}
\date{18 April 2022}

\begin{document}
	\maketitle
	
	\section{Compact operators}
	References: Murphy (\emph{$C^*$ algebras and operator theory}), Br\'{e}zis, Daners
	
	Notation: for a normed vector space $E$, we write $B_E$ for the closed unit ball $\{u\in E : \|u\| \le 1\}$ in $E$.
	
	\begin{definition}
		Let $E,F$ be Banach spaces. A linear operator $T:E\to F$ is called \emph{compact} if $T(B_E)$ has compact closure in $F$. (Equivalently, we say that $T(B_E)$ is relatively compact in $F$).
	\end{definition}
	
	\begin{remark}
		\begin{enumerate}[(i)]
			\item By linearity, $T$ is compact if and only if $T(B)$ has compact closure in $F$ for all bounded subsets $B\subset E$. 
			\item Moreover, by the characterisation of compactness in metric spaces, $T$ is compact if and only if for all bounded sequences $(x_n)_{n\in\NN} \subset E$, $(Tx_n)_{n\in\NN}$ has a convergent subsequence in $F$, if and only if $T(B)$ is totally bounded for all bounded subsets $B\subset E$.
			\item Since compact metric spaces are separable, it also follows that $T$ has compact range.
			\item It is clear that finite-rank operators are compact.
		\end{enumerate}
	\end{remark}
	
	\begin{theorem}
		The set $\mathcal{K}(E,F)$ of compact operators from $E$ to $F$ is a closed linear subspace of $\mathcal{L}(E,F)$. In particular, if $T\in\mathcal{L}(E,F)$ is the uniform limit of a sequence of finite rank operators $(T_n)_{n\in\NN}$ (i.e.\ $\|T_n-T\|_{\mathcal{L}(E,F)} \to 0$), then $T\in\mathcal{K}(E,F)$.
	\end{theorem}
	
	\begin{proposition}
		Let $E,F,G$ be Banach spaces. If $T\in\mathcal{L}(E,F)$ and $S\in\mathcal{K}(F,G)$, then $S\circ T\in\mathcal{K}(E,G)$. Similarly, if $T\in\mathcal{K}(E,F)$ and $S\in\mathcal{L}(F,G)$, then also $S\circ T\in\mathcal{K}(E,G)$. In particular, $\mathcal{K}(E)$ is a closed, two-sided ideal in the algebra $\mathcal{L}(E)$.
	\end{proposition}
	
	\begin{theorem}[Schauder]
		Let $E,F$ be Banach spaces. Then $T\in\mathcal{K}(E,F)$ if and only if $T^*\in\mathcal{K}(F^*,E^*)$.
	\end{theorem}
	\begin{proof}
		Assume that $T\in\mathcal{K}(E,F)$, so that $T(B_E)$ is totally bounded in $F$. Given $\varepsilon >0$, there exist elements $x_1,\ldots,x_n\in B_E$ such that if $x\in B_E$, then $\|Tx-Tx_i\| < \varepsilon/3$ for some $i=1,\ldots,n$. Now define a linear map $v:F^*\to\CC^n$ by
		\begin{equation*}
			v(\psi) := (\braket{\psi,Tx_1}, \ldots, \braket{\psi,Tx_n}), \qquad \psi\in F^*
		\end{equation*}
		where we equip $\CC^n$ with the supremum norm $\|u\|=\max_{1\le i\le n}|u_i|$. Since $v$ has finite rank, $v$ is a compact operator, and hence $v(B_{F^*})$ is totally bounded in $\CC^n$. Consequently there exist functionals $\psi_1,\ldots,\psi_m\in B_{F^*}$ such that if $\psi\in B_{F^*}$, then
		\begin{equation*}
			\|v(\psi)-v(\psi_k)\| = \max_{1\le j\le n}|\braket{T^*\psi,x_j}-\braket{T^*\psi_k,x_j}| < \frac{\varepsilon}{3}
		\end{equation*}
		for some $k=1,\ldots,m$. By the above arguments, we find
		\begin{align*}
			|\braket{T^*\psi,x}-\braket{T^*\psi_k,x}| &\le |\braket{T^*\psi,x}-\braket{T^*\psi,x_i}| + |\braket{T^*\psi,x_i}-\braket{T^*\psi_k,x_i}| \\
			&\qquad + |\braket{T^*\psi_k,x_i}-\braket{T^*\psi_k,x}| \\
			&\le 2\|Tx-Tx_i\| + \|v(\psi)-v(\psi_k)\| < 3 \cdot\frac{\varepsilon}{3}=\varepsilon
		\end{align*}
		for all $x\in B_E$. Thus $\|T^*\psi -T^*\psi_k\|\le\varepsilon$ for some $k=1,\ldots,m$, which shows that $T^*(B_{F^*})$ is totally bounded in $E$. Hence $T^*:F^*\to E^*$ is a compact operator.
		
		Now assume $T^*\in\mathcal{K}(F^*,E^*)$. Using what we proved above, we know that $T^{**}:E^{**}\to F^{**}$ is compact. By abuse of notation, we will write $j$ for the canonical embeddings $E\to E^{**}$ and $F\to F^{**}$. Recall that $j$ is an isometry (a consequence of the Hahn-Banach theorem). By assumption, $T^{**}(j(B_E))$ is relatively compact in $F^{**}$. Observe that $j(T(B_E) \subseteq j(F)$ where $j(F)$ is a closed subspace of $F^{**}$ isomorphic to $F$, and moreover it is easy to verify that $T^{**}(j(B_E)) = j(T(B_E))$. Thus $j(T(B_E))$ is compact, which implies that $T(B_E)$ is relatively compact in $F$.
	\end{proof}
	
	\begin{lemma}[Riesz]
		\label{lem:Riesz}
		Let $E$ be a normed vector space, and suppose $M$ is a proper, closed linear subspace. Then for every $\varepsilon >0$ there exists $u\in E$ with $\|u\|=1$ and $\dist(u,M)\ge 1-\varepsilon$.
	\end{lemma}
	
	\begin{theorem}[Riesz]
		\label{thm:compact-ball}
		Let $E$ be a normed vector space such that $B_E$ is compact. Then $\dim E < \infty$.
	\end{theorem}
	\begin{proof}
		Assume by contradiction that $\dim E = \infty$. Then there is a sequence of finite dimensional subspaces $E_n$ such that $E_n \subset E_{n+1}$ and $E_n \ne E_{n+1}$ for all $n\in\NN$. By Lemma~\ref{lem:Riesz}, there exists a sequence $(u_n)_{n\in\NN}$ with $u_n\in E_n$, $\|u_n\|=1$, and $\dist(u_n,E_{n-1})\ge \frac{1}{2}$. In particular, $\|u_n-u_m\|\ge \frac{1}{2}$ if $m<n$, and thus it follows that $(u_n)_{n\in\NN}$ has no convergent subsequence. This contradicts the compactness of $B_E$.
	\end{proof}
	
	\begin{theorem}
		\label{thm:Riesz-Fredholm}
		Let $T\in\mathcal{K}(E)$ and $\lambda\in\CC\setminus\{0\}$. Then for all $k\in\NN$,
		\begin{enumerate}[\upshape (i)]
			\item $\dim(\ker(\lambda I-T)^k) < \infty$, and
			\item $\rg(\lambda I-T)^k$ is closed in $E$.
			\item Moreover, $\rg(\lambda I-T)$ has finite codimension in $E$, and
			\begin{equation*}
				\dim(E/\rg(\lambda I-T)) = \dim(\ker(\lambda I-T^*)).
			\end{equation*}
		\end{enumerate}
	\end{theorem}
	\begin{proof}
		Since $\ker(\lambda I-T)^k = \ker(I-\lambda^{-1}T)^k$ and similarly $\rg(\lambda I-T)^k = \rg(I-\lambda^{-1}T)^k$, we may assume $\lambda=1$ without loss of generality (replacing $T$ by $\lambda T$). Moreover, it suffices to prove the theorem for $k=1$. Indeed, we have
		\begin{equation*}
			(I-T)^k = \sum_{j=0}^k \binom{k}{j}(-1)^j T^j = I - \tilde{T},
		\end{equation*}
		where $\tilde{T} := \sum_{j=1}^k (-1)^{j-1}\binom{k}{j}T^j$ is a compact operator.
		
		(i): Firstly, note that $Z=\ker(I-T)$ is a Banach space, since it is a closed subspace of $E$. If $x\in Z$, then $Tx=x$, and
		\begin{equation*}
			S := \{x\in Z : \|x\|=1\} \subseteq \overline{T(B_E)}.
		\end{equation*}
		Since $T\in\mathcal{K}(E)$, it follows that $S$ is a compact subset of $E$, and hence $\dim Z < \infty$.
	\end{proof}
	
	
\end{document}
