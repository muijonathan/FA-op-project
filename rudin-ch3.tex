\begin{exercise}
	A set $H \subset \RR^n$ is called a \emph{hyperplane} if there exists a non-zero vector $a = (a_1, \ldots, a_n)$ and $c \in \RR$ such that
	\begin{equation*}
		H = \{ x \in \RR^n : a \cdot x = c\}.
	\end{equation*}
	Suppose $E$ is a convex set in $\RR^n$ with non-empty interior, and $y$ is a boundary point of $E$. Prove that there is a hyperplane $H$ such that $y\in H$ and $E$ lies entirely on one side of $H$ (make this statement precise). [\emph{Hint}: suppose $0 \in \text{int}(E)$, let $M = \text{span}\{y\}$ and then use Hahn-Banach].
\end{exercise}

\begin{exercise}
	Consider $L^2 = L^2([-1,1])$ with Lebesgue measure. For each scalar $\alpha$, let $E_\alpha$ be the set of all continuous functions $f$ on $[-1,1]$ such that $f(0) = \alpha$. Show that each $E_\alpha$ is convex and that each is dense in $L^2$. Hence, if $\alpha \ne \beta$, $E_\alpha$ and $E_\beta$ are disjoint convex sets which cannot be separated by any continuous linear functional $\Lambda$. [\emph{Hint}: what is $\Lambda(E_\alpha)$?].
\end{exercise}

\begin{exercise}[Ex.\ 4]
	Let $\ell^\infty(\NN)$ be the space of bounded real-valued functions on $\NN$ (i.e.\ sequences), and define the translation operator $\tau$ on $\ell^\infty$ by $(\tau x)(n) = x(n+1)$. Prove that there exists a linear functional $\Lambda$ on $\ell^\infty$ such that
	\begin{enumerate}[(a)]
		\item $\Lambda (\tau x) = \Lambda x$; and
		\item $\displaystyle\liminf_{n\to\infty} x(n) \le \Lambda x \le \displaystyle\limsup_{n\to\infty} x(n)$
	\end{enumerate}
	for every $x \in \ell^\infty$. Such a linear functional is called a \emph{Banach limit}.
	
	[\emph{Hints}: Define
	\begin{align*}
		\Lambda_n x &= \frac{x(1)+\ldots+x(n)}{n} \\
		M &= \{ x \in \ell^\infty : \lim_{n\to\infty} \Lambda_n x =: \Lambda x \text{ exists}\} \\
		p(x) &= \limsup_{n\to\infty} \Lambda_n x
	\end{align*}
	and use Hahn-Banach].
\end{exercise}
\textcolor{red}{The definition of $p(x)$ appears to be useless, see correction below}.

\begin{proof}[Solution]
	Let $x \in M$. Firstly we verify that $\Lambda$ is translation invariant on $M$. Observe that
	\begin{align*}
		\Lambda_n(\tau x) &= \frac{x(2)+\ldots+x(n+1)}{n} \\
		&= \frac{x(1)+\ldots+x(n)}{n} + \frac{x(n+1)-x(1)}{n} \\
		&= \Lambda_n x + \frac{x(n+1)-x(1)}{n}.
	\end{align*}
	As $x$ is a bounded sequence, we have $[x(n+1)-x(1)]/n \to 0$ as $n\to\infty$, and thus the limit on the right hand side exists and equals $\Lambda x$. Hence $\lim_{n\to\infty} \Lambda_n (\tau x) = \Lambda x$ as well. 
	
	Now define
	\begin{equation*}
		p(x) := \limsup_{n\to\infty} x(n).
	\end{equation*}
	We have $p(x+y) \le p(x)+p(y)$ and $p(tx) = tp(x)$ for all $x,y \in \ell^\infty$ and $t>0$ by elementary properties of the limit superior. For $x \in M$ and for each $n\in\NN$, let $i_n \in \NN$ be such that $x(i_n) = \max_{1\le k \le n} x(k)$. Then $(i_n)_{n\in\NN}$ is a non-decreasing sequence, and
	\begin{equation*}
		\Lambda_n x = \frac{x(1)+\ldots+x(n)}{n} \le x(i_n) \le \sup_{k \ge i_n} x(k).
	\end{equation*}
	Passing to the limit, we obtain $\Lambda x \le \limsup_{n\to\infty}x(n) = p(x)$ for all $x \in M$. Thus, by the Hahn-Banach theorem, $\Lambda$ admits an extension (still denoted by $\Lambda$) to $\ell^\infty$ such that $-p(-x) \le \Lambda x \le p(x)$ for all $x \in \ell^\infty$, i.e.
	\begin{equation*}
		\liminf_{n\to\infty} x(n) \le \Lambda x \le \limsup_{n\to\infty} x(n),
	\end{equation*}
	which proves property (b). To show (a), we can apply the above inequality to $\tau x$ instead of $x$, which yields
	\begin{equation*}
		\liminf_{n\to\infty} (\tau x)(n) \le \Lambda(\tau x) \le \limsup_{n\to\infty} (\tau x)(n).
	\end{equation*}
	Since $\sup_{k \ge n} (\tau x)(k) = \sup_{k \ge n} x(k+1) \le \sup_{k \ge n}x(k)$, we find that $\Lambda (\tau x) \le \limsup_{n\to\infty} x(n)$. Hence $\Lambda(\tau x) - \Lambda x \le 0$. Similarly, by considering the lower bounds, we find that $\Lambda(\tau x) - \Lambda x \ge 0$, and by combining the two cases we conclude $\Lambda (\tau x) = \Lambda x$. This completes the proof.
\end{proof}

\begin{exercise}[Ex.\ 5(b),(c)]
	\begin{enumerate}[(i)]
		\item Consider the sequence space $\ell^p$ with $1 < p < \infty$. Show that $\ell^p$ contains sequences which converge weakly but not strongly.
		\item On the other hand, prove that every weakly convergent sequence in $\ell^1$ converges strongly, despite the fact that the weak topology is different from the strong topology (induced by the norm).
	\end{enumerate}
\end{exercise}

\subsection*{Weak convergence in $\ell^1$}
We follow Problem 8 in the functional analysis book of Br\'{e}zis.

Let $E = \ell^1$, so that $E^\star = \ell^\infty$ as is well-known. Let $(x^{(n)})_{n\in\NN}$ be a sequence in $E$ that converges weakly to 0. The goal is to show that $\|x^{(n)}\|_1 \to 0$, i.e.\ the sequence in fact converges strongly.

\begin{enumerate}
	\item Given $f, g \in B_{E^\star}$, i.e.\ the unit ball in $\ell^\infty$, define
	\begin{equation*}
		d(f,g) := \sum_{i=1}^\infty \frac{1}{2^i}|f_i - g_i|.
	\end{equation*}
	Check that $d$ is a metric on $B_{E^\star}$ and that $B_{E^\star}$ is compact for the corresponding topology.
	
	\item Given $\varepsilon >0$, set
	\begin{equation*}
		F_k := \{ f \in B_{E^\star} : |f(x^{(n)})| \le \varepsilon \text{ for all } n \ge k\}.
	\end{equation*}
	Prove that there exists some $f^0 \in B_{E^\star}$, a constant $\rho >0$ and a $k_0 \in\NN$ such that
	\begin{equation*}
		f \in B_{E^\star} \text{ and } d(f,f^0) < \rho \implies f \in F_{k_0}.
	\end{equation*}
	[\emph{Hint}: Baire category theorem].
	
	\item Fix $N\in\NN$ such that $1/2^{N-1} < \rho$. Prove that
	\begin{equation*}
		\|x^{(n)}\|_1 \le \varepsilon + 2 \sum_{i=1}^N |x_i^{(n)}| \qquad \text{for all } n \ge k_0
	\end{equation*}
	and hence conclude.
\end{enumerate}