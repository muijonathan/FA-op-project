\subsection{A reminder of the separation axioms}
\begin{definition}
	A topological space $X$ is called
	\begin{enumerate}[(i)]
		\item $T_1$ if every singleton $\{x\}$ is closed; equivalently, for every pair of distinct points $x,y \in X$, there is an open set $U$ such that $x \in U$ and $y \not\in U$.
		\item $T_2$ or \emph{Hausdorff} if for every pair of distinct points $x,y \in X$, there are disjoint open sets $U_1, U_2$ such that $x \in U_1, y \in U_2$.
		\item \emph{regular} if for every $x \in X$ and every non-empty closed set $A \subset X$ with $x \not\in A$, there exist disjoint open sets $U_1, U_2$ such that $x \in U_1, A \subset U_2$.
		\item $T_3$ if it is both $T_1$ and regular.
		\item \emph{normal} if for every pair of disjoint, non-empty closed subsets $A, B \subset X$, there exist disjoint open sets $U_1, U_2$ such that $A \subset U_1, B \subset U_2$.
		\item $T_4$ if it is both $T_1$ and normal.
	\end{enumerate}
\end{definition}

Clearly $T_4 \Rightarrow T_3 \Rightarrow T_2 \Rightarrow T_1$. A non-empty set $X$ equipped with the trivial (a.k.a.\ indiscrete) topology $\tau = \{ \emptyset, X \}$ is normal (since the definition is satisfied vacuously) but clearly not $T_1$. \emph{Warning}: in some texts (e.g.\ Reed \& Simon), the definitions of `regular' and `normal' include the $T_1$ axiom (which has the advantage of excluding trivial examples like the indiscrete topology).

\begin{proposition}
	A topological space $X$ is regular if and only if for every $x \in X$, every open neighbourhood of $x$ contains the closure of another neighbourhood of $x$.
\end{proposition}

\begin{proof}
	Assume that $X$ is regular, and fix $x \in X$. Let $U$ be any neighbourhood of $x$. Then $U^c$ is a closed set that does not contain $x$. By regularity, there are disjoint open sets $V, W$ such that $x \in V$ and $U^c \subset W$. Since $W^c$ is closed, it follows that $\overline{V} \subseteq \overline{W}^c = W^c \subset U$.
	
	Now we prove the converse. Fix $x \in X$ and let $A$ be a non-empty closed set not containing $x$. By assumption, every neighbourhood of $x$ contains the closure of another neighbourhood of $x$. Since $A^c$ is a neighbourhood of $x$, there exists another neighbourhood $U$ such that $x \in U \subset \overline{U} \subset A^c$. This implies $A \subset \overline{U}^c$, and thus $U$ and $\overline{U}^c$ are disjoint open sets containing $x$ and $A$ respectively. Thus $X$ is regular.
\end{proof}

With a very similar proof, one shows the following.
\begin{proposition}
	A topological space $X$ is normal if and only if every neighbourhood of a closed subset $A$ contains the closure of another neighbourhood of $A$.
\end{proposition}

\begin{proposition}
	Let $K$ be a compact subset of a Hausdorff space. For every $x \in K^c$, there exist disjoint open subsets $U, V$ such that $x \in U$ and $K \subset V$. Consequently, every compact subset of a Hausdorff space is closed.
\end{proposition}

\begin{proof}
	Fix $x \in K^c$. For every $y \in K$, there exist disjoint neighbourhoods $U_y$ of $y$ and $V_y$ of $x$, since $X$ is Hausdorff. Then $K$ is covered by the family $\{V_y\}_{y \in K}$, and by compactness, there exist finitely many points $y_1, \ldots, y_n$ such that $K \subseteq V := \bigcup_{i=1}^n V_{y_i}$. One verifies easily that $U := \bigcap_{i=1}^n U_{y_i}$ is an open set containing $x$ that is disjoint from $V$.
\end{proof}

Here are some other important facts.
\begin{proposition}[Source: Reed \& Simon, IV Exercise\ 4]
	A topological space is called \emph{Lindel\"{o}f} if every open cover has a countable subcover.
	\begin{enumerate}[(a)]
		\item Every second countable space is Lindel\"{o}f.
		\item Every second countable $T_3$ space is $T_4$.
	\end{enumerate}
\end{proposition}

\begin{proposition}
	Every metric space is a $T_4$ space.
\end{proposition}

\begin{proposition}
	Every topological vector space $X$ is regular.
\end{proposition}

\begin{proof}
	Let $x \in X$ be a closed set, and suppose $x \not\in C$. The proposition follows by applying Rudin Theorem 1.10 with the compact set $K = \{x\}$.
\end{proof}

\begin{theorem}[Source: Rudin RCA, Theorem 2.7]
	Suppose $U$ is an open subset of a locally compact Hausdorff space $X$. If $K \subset U$ and $K$ is compact, then there is an open set $V$ such that $K \subset V \subset \overline{V} \subset U$.
\end{theorem}

\begin{corollary}
	Every locally compact Hausdorff space is $T_3$, and every compact Hausdorff space is $T_4$.
\end{corollary}

\subsection{Rudin Chapter 2 Exercises}

\begin{exercise}
	If $X$ is an infinite-dimensional tvs which is the union of countably many finite-dimensional subspaces, prove that $X$ is of the first category. Hence show that no infinite dimensional $F$-space has a countable Hamel basis.
\end{exercise}

\begin{proof}[Solution]
	Firstly, note that in any tvs $X$, if a non-trivial subspace $Y$ has \emph{non}-empty interior, then $Y = X$. Indeed, suppose there exists $x \in Y$ such that $x+U \subset Y$ for some open neighbourhood $U$ of 0. Since $Y$ is a subspace, $U = -x + (x+U)$ then belongs to $Y$. Consequently $nU \subset Y$ for all $n \in \NN$. However, by Proposition 1.15(a), $X = \bigcup_{n=1}^\infty nU$, hence $Y=X$ as claimed.
	
	Now assume that the tvs $X$ is the countable union of finite-dimensional subspaces $Y_n, n \in \NN$. Of course we may assume that $Y_n \ne \{0\}$ for all $n \in \NN$. Then each $Y_n$ is a proper non-trivial subspace of $X$. Recall from Theorem 1.21(b) that any finite-dimensional subspace of a tvs is closed. By the argument in the previous paragraph, we deduce $\text{int}(\overline{Y_n}) = \text{int}(Y_n) = \emptyset$ for all $n \in \NN$, i.e.\ each $Y_n$ is nowhere dense in $X$. This means that $X$ is of the first category.
	
	If $X$ has a countable Hamel basis $\{x_n\}_{n \in \NN}$, consider $Y_n := \text{span}\{x_1, \ldots, x_n\}$. Obviously each $Y_n$ is finite-dimensional, and $X = \bigcup_{n=1}^\infty Y_n$. Therefore $X$ is of the first category. Since an $F$-space is a complete metrizable tvs, it follows from the Baire category theorem that no $F$-space can have a countable Hamel basis.
\end{proof}

\begin{exercise}[Rudin Ex.\ 2.4]
	Let $L^1, L^2$ denote the usual Lebesgue spaces on the unit interval $(0,1)$. Prove that $L^2$ is of the first category in $L^1$ in three ways:
	\begin{enumerate}[(a)]
		\item Show that the set $\{ f\in L^1 : \int |f|^2 \le n\}$ is closed in $L^1$ but has empty interior.
		\item Set $g_n := n \mathbf{1}_{[0, n^{-3}]}$ and show that
		\begin{equation*}
			\int fg_n \longrightarrow 0
		\end{equation*}
		as $n \to \infty$ for every $f\in L^2$, but not for every $f\in L^1$.
		\item Note that the inclusion $L^2 \hookrightarrow L^1$ is continuous but not onto.
	\end{enumerate}
	Do the same exercise for $L^p, L^q$ where $p<q$.
\end{exercise}

\begin{proof}
	(a): Let $(f_k)_{k \in \NN}$ be a sequence in $Y_n$ such that $f_k \to f$ in $L^1$. By a standard result in functional analysis, there is a subsequence, still denoted by $(f_k)$, that converges almost everywhere to $f$. Hence $|f_k|^2 \to |f|^2$ almost everywhere, and $\int |f_k|^2 \le n$. By the dominated convergence theorem, it follows that $\int |f|^2 \le n$, so $f \in Y_n$. This shows that $Y_n$ is closed in $L^1$.
	
	To see that $Y_n$ has empty interior, let $\varepsilon > 0$ and $f \in Y_n$ be given. Notice that the function $x^{-1/2}$ belongs to $L^1(0,1)$ but not to $L^2(0,1)$, since $1/x$ is not integrable in a neighbourhood of 0. Since
	\begin{equation*}
		\int_0^t \frac{1}{2\sqrt{x}} \,dx = \sqrt{t}, \qquad t \in (0,1),
	\end{equation*}
	we define the function
	\begin{equation*}
		g(x) = \begin{cases}
			\frac{1}{2\sqrt{x}} \qquad &x \in (0, \tfrac{\varepsilon^2}{4}) \\
			f(x) \qquad &x \in [\tfrac{\varepsilon^2}{4}, 1).
		\end{cases}
	\end{equation*}
	Now fix $s > 0$ small enough such that $\int_0^s |f(x)| \,dx < \varepsilon/2$. Then for $0 < r < \min\{\varepsilon^2/4, s\}$, we have
	\begin{align*}
		\|g-f\|_1 &\le \|g - \mathbf{1}_{[r,1)}f\|_1 + \|\mathbf{1}_{[r,1)}f -f\|_1 \\
		&< \int_0^r \frac{1}{2\sqrt{x}} \,dx + \int_0^r |f(x)|\,dx \\
		&< \frac{\varepsilon}{2} + \frac{\varepsilon}{2} = \varepsilon.
	\end{align*}
	However $g$ does not belong to any $Y_n$, which shows that each $Y_n$ has empty interior. Finally, it is clear that $L^2(0,1) = \bigcup_{n\in\NN} Y_n$, and hence $L^2(0,1)$ is of the first category in $L^1(0,1)$.
	
	(b): Define the linear maps $\Lambda_n : L^1(0,1) \to \CC$ by
	\begin{equation*}
		\Lambda_n f := \int fg_n \,dx.
	\end{equation*}
	Then $|\Lambda_n f| \le \|f\|_1 \|g_n\|_\infty = n\|f\|_1$, so $\Lambda_n$ is continuous for every $n \in \NN$. Now if $f \in L^2(0,1)$, then
	\begin{equation*}
		|\Lambda_n f| \le \|f\|_2 \|g_n\|_2 = \frac{1}{\sqrt{n}}\|f\|_2 \longrightarrow 0
	\end{equation*}
	as $n \to \infty$. Thus $L^2(0,1)$ is a subspace of $C := \{ f\in L^1(0,1): \lim_{n\to\infty} \Lambda_n f \text{ exists}\}$. By Theorem 2.7, if $C$ is of second category, then $C=L^1(0,1)$. We show that this is not the case. Consider the function $f(x) = x^{-5/6}$, and note that $f \in L^1(0,1)$. However,
	\begin{equation*}
		\Lambda_n f = n\int_0^{1/n^3} x^{-5/6} \,dx = 6n \cdot \frac{1}{n^{1/2}} = 6\sqrt{n} \longrightarrow \infty,
	\end{equation*}
	so clearly $f \not\in C$. Thus $L^2$ is of the first category in $L^1$.
	
	(c): A simple application of Cauchy-Schwarz yields
	\begin{equation*}
		\int |f| \,dx = \int 1 \cdot |f| \,dx \le \|f\|_2
	\end{equation*}
	so the inclusion $L^2 \hookrightarrow L^1$ is continuous. However, $f(x) = x^{-1/2}$ belongs to $L^1$ but not to $L^2$, so the inclusion is not onto. It follows from the open mapping theorem that if $L^2(0,1)$ is of second category in $L^1(0,1)$, then the inclusion map would be surjective, which is false. Hence $L^2$ is of first category in $L^1$.
\end{proof}

\begin{exercise}[Rudin Ex.\ 2.5]
	Prove results analogous to Exercise 4 for the sequence spaces $\ell^p, \ell^q$.
\end{exercise}

\begin{exercise}[Rudin Ex.\ 2.6]
	Recall that the Fourier coefficients $\hat{f}(k)$ of a function $f \in L^2(\mathbb{T})$ are given by
	\begin{equation*}
		\hat{f}(k) = \frac{1}{2\pi}\int_{-\pi}^\pi f(e^{i\theta}) e^{-ik\theta} \,d\theta, \qquad k \in \ZZ.
	\end{equation*}
	Put
	\begin{equation*}
		\Lambda_n f := \sum_{k=-n}^n \hat{f}(k), \qquad n \in \NN.
	\end{equation*}
	Prove that $Y = \{f \in L^2(\mathbb{T}) : \lim_{n\to\infty} \Lambda_n f \text{ exists}\}$ is a dense subspace of $L^2(\mathbb{T})$ of the first category.
\end{exercise}

\begin{proof}
	We prove the density first. Define $e_k(x) := e^{ikx}$ for every $x \in (-\pi,\pi)$ and $k \in \ZZ$. It is a standard fact in analysis that the set $\{e_k\}_{k \in \ZZ}$ forms an orthonormal basis of $L^2(\mathbb{T})$, and thus every $f \in L^2(\mathbb{T})$ has the representation
	\begin{equation*}
		f(x) = \sum_{k \in \ZZ} \braket{f,e_k}_{L^2} e_k(x) = \sum_{k \in \ZZ} \hat{f}(k) e^{ikx}, \qquad x \in (-\pi,\pi)
	\end{equation*}
	and the series converges in $L^2$. In other words, the functions $f_m$ defined by the partial sums
	\begin{equation*}
		f_m(x) = \sum_{k=-m}^m \hat{f}(k) e_k(x), \qquad m \in \NN
	\end{equation*}
	converge in $L^2(\mathbb{T})$ to $f$. Observe that for a fixed $m\in\NN$, the Fourier coefficients of $f_m$ are precisely those of $f$ when $|k|\le m$, and $\hat{f}_m(k)=0$ whenever $|k|>m$. It follows that
	\begin{equation*}
		\lim_{n\to \infty} \Lambda_n f_m = \sum_{k=-\infty}^\infty \hat{f}_m(k) = \sum_{k=-m}^m \hat{f}(k) < \infty
	\end{equation*}
	which shows that $f_m \in Y$ for all $m\in\NN$. Since $f_m \to f$ in $L^2(\mathbb{T})$, we deduce that $Y$ is a dense subspace.
	
	Now we prove that $Y$ is of the first category in $L^2(\mathbb{T})$. To do so, we make use of the \emph{Dirichlet kernel}
	\begin{equation*}
		D_n(x) := \sum_{k=-n}^n e^{ikx} = \frac{\sin ((n+\tfrac{1}{2})x)}{\sin \tfrac{x}{2}} \qquad n \in \NN, x \in (-\pi,\pi).
	\end{equation*}
	(note that $D_n(0) = 2n+1$). Now we observe
	\begin{equation*}
		\Lambda_n f = \sum_{k=-n}^n \hat{f}(k) = \frac{1}{2\pi} \int_{-\pi}^\pi f(e^{i\theta}) \sum_{k=-n}^n e^{-ik\theta} \,d\theta = \frac{1}{2\pi}\int_{-\pi}^\pi f(e^{i\theta}) D_n(\theta) \,d\theta,
	\end{equation*}
	i.e.\ $\Lambda_n f$ is exactly the inner product of $f$ with $D_n$. Moreover, for every $n\in \NN$, $\Lambda_n : L^2(\mathbb{T}) \to \CC$ is a bounded linear operator. Indeed, using the Cauchy-Schwarz inequality and (ii) from the Lemma below, we find
	\begin{equation*}
		|\Lambda_n f| \le \|f\|_2 \|D_n\|_2 = (2n+1)\|f\|_2
	\end{equation*}
	for all $f \in L^2(\mathbb{T})$. Since $\CC$ is complete, the set $Y$ is the same as the set of all $f \in L^2(\mathbb{T})$ for which $(\Lambda_n f)_{n\in\NN}$ is a Cauchy sequence. We now show that $Y \ne L^2(\mathbb{T})$, and then Theorem 2.7(a) will imply (by contrapositive) that $Y$ is \emph{not} of the second category, i.e.\ $Y$ is of the first category in $L^2$. By the uniform boundedness principle, it suffices to show that the family of maps $\{\Lambda_n\}_{n \in \NN}$ is not uniformly bounded. We show that there exists a bounded sequence $(g_n) \subset L^2(\mathbb{T})$ such that $|\Lambda_n g_n| \to \infty$ as $n\to\infty$. Consider the functions
	\begin{equation*}
		g_n(x) = \text{sgn}(D_n(x)).
	\end{equation*}
	(Recall the signum function $\text{sgn}(t)$ takes the value 1 when $t>0$, $-1$ when $t<0$ and 0 for $t=0$). Then $|g_n(x)|^2 = 1$ for almost all $x\in(-\pi,\pi)$, since each $D_n$ has only finitely many zeroes in the interval. Thus $\|g_n\|_{L^2} = 1$ for all $n\in\NN$, so $(g_n)$ is a bounded sequence. However,
	\begin{equation*}
		|\Lambda_n g_n| = \frac{1}{2\pi}\int_{-\pi}^\pi \text{sgn}(D_n(x))D_n(x) \,dx = \frac{1}{2\pi}\int_{-\pi}^\pi |D_n(x)|\,dx \to \infty
	\end{equation*}
	by (iii) in the Lemma below. Thus $\{\Lambda_n\}_{n\in\NN}$ is not uniformly bounded, and the proof is complete.
\end{proof}

\begin{lemma}[Standard facts about the Dirichlet kernel]
	The Dirichlet kernel satisfies
	\begin{enumerate}[\upshape (i)]
		\item $\displaystyle\frac{1}{2\pi}\displaystyle\int_{-\pi}^{\pi} D_n(x) \,dx = 1$ for every $n\in\NN \cup \{0\}$;
		\item $\displaystyle\frac{1}{2\pi}\displaystyle\int_{-\pi}^{\pi} |D_n(x)|^2 \,dx = 2n+1$ for every $n\in\NN \cup \{0\}$; and
		\item $\displaystyle\frac{1}{2\pi}\displaystyle\int_{-\pi}^{\pi} |D_n(x)| \,dx \ge \frac{2}{\pi^2} \sum_{k=1}^n \frac{1}{k}$ for every $n\in\NN$.
	\end{enumerate}
	In particular $\|D_n\|_{L^1(\mathbb{T})} \to \infty$ and $\|D_n\|_{L^2(\mathbb{T})} \to \infty$ as $n\to\infty$.
\end{lemma}

\begin{proof}
	(i): Fix $n \ge 0$, and recall the trigonometric identity $\sin A - \sin B = 2\sin(\tfrac{A-B}{2})\cos(\tfrac{A+B}{2})$. Then
	\begin{equation*}
		\sin((n+1+\tfrac{1}{2})x) - \sin((n+\tfrac{1}{2})x) = 2\sin(\tfrac{x}{2})\cos((n+1)x),
	\end{equation*}
	which yields
	\begin{equation*}
		\frac{1}{2\pi}\int_{-\pi}^\pi D_{n+1}(x) - D_n(x) \,dx = \frac{1}{2\pi}\int_{-\pi}^\pi 2 \cos((n+1)x) \,dx = 0
	\end{equation*}
	which implies that the integral of $D_n$ over $(-\pi,\pi)$ has the same value for all $n\ge 0$. Hence
	\begin{equation*}
		\frac{1}{2\pi}\int_{-\pi}^\pi D_n(x) \,dx = \frac{1}{2}\int_{-\pi}^\pi D_0(x) \,dx = 1
	\end{equation*}
	for all $n \ge 0$ as claimed.
	
	(ii): We write $\braket{\cdot, \cdot}$ for the inner product on $L^2(\mathbb{T})$. Using the orthonormality of the set $\{e_k\}_{k\in\ZZ}$ where $e_k(x) = e^{ikx}$, we obtain
	\begin{equation*}
		\frac{1}{2\pi}\int_{-\pi}^\pi |D_n(x)|^2 \,dx = \braket{D_n, D_n} = \sum_{k,\ell = -n}^n \braket{e_k,e_\ell} = \sum_{k,\ell=-n}^n \delta_{k\ell} = 2n+1.
	\end{equation*}
	
	(iii): Recall the elementary estimate $\sin t \le t$ valid for all $t\ge 0$. Now we compute
	\begin{align*}
		\int_{-\pi}^\pi |D_n(x)|\,dx = 2\int_0^\pi |D_n(x)|\,dx &= 2\int_0^\pi \frac{|\sin((n+\tfrac{1}{2})x)|}{\sin(\tfrac{x}{2})} \,dx \\
		&\ge 4\int_0^\pi \frac{|\sin((n+\tfrac{1}{2})x)|}{x} \,dx \\
		[t = (n+\tfrac{1}{2})x] \quad &= 4\int_0^{(n+1/2)\pi} \frac{|\sin t|}{t} \,dt \\
		&\ge 4\int_0^{n\pi} \frac{|\sin t|}{t}\,dt \\
		&= 4\sum_{k=1}^n \frac{1}{k\pi} \int_{(k-1)\pi}^{k\pi} |\sin t|\,dt = \frac{4}{\pi} \sum_{k=1}^n \frac{1}{k}
	\end{align*}
	which completes the proof.
\end{proof}

\begin{remark}
	As another application, we note that the Dirichlet kernel and the uniform boundedness principle is used to prove the following theorem due to du Bois Reymond: there exists a continuous function $f$ on $\mathbb{T}$ whose partial Fourier sums diverge at $x=0$. [Reference: Grafakos, \emph{Classical Fourier Analysis}, Section 3.4.3].
\end{remark}